\section{РУКОВОДСТВО ПОЛЬЗОВАТЕЛЯ}
\label{sec:manual}

Модуль выделения информационных признаков из музыкального произведения представляет собой API на языке программирования Python, который решает задачи выделение спектральных, временных и иных признаков из музыкального трека, проверка значимости признаков путём использовании их в задаче жанровой классификации, визуализация данных алгоритмом t-SNE.


\subsection{Требования к аппартному и программному обеспечению}
\label{sub:manual:sys}

Минимальные требования для полнооценного функционирования программного обеспечения:

\begin{itemize}
\item операционные системы Windows XP с пакетом обновления 2 +, Windows Vista, Windows 7, Windows 8, Windows 10, Mac OS X 10.6 или более поздней версии Ubuntu 10.04 +, Debian 6 +, OpenSuSE 11.3 +, Fedora Linux 14;
\item установленный интерпретатор Python 2.7;
\item пакетный менеджер pip;
\item pip-пакет Scipy версии 19.0 +;
\item pip-пакет scikit-learn версии 0.18.1 +; 
\item pip-пакет PeakUtils  версии 1.1.0 +;
\item pip-пакет scilits.talkbox версии 0.2.5+;
\item pip-пакет pymongo;
\item установленный MongoDB;
\item процессор Intel Pentium 4 / Athlon 64 или более поздней версии с поддержкой SSE2;
\item свободное место на жестком диске 350 Мб;
\item оперативная память 512 Mб.
\end{itemize}


\subsection{Руководство по установке и запуску программного средства}
\label{sub:manual:install}

Для установки приложения необходимо запустить установщик Setup.exe на любом жестком диске, в котором будет достаточно свободного пространства. Требования описаны выше. В результате этого создастся папка, в которой будут находиться все скрипты нужные для функционирования модуля. Для использования модуля необходимо добавить скрипты в собственный проект и с помощью команды \texttt{import} подключить \texttt{MainModule} для выделения информационных образов, \texttt{GenreClassificationModule} для жанровой классификации и  \texttt{VisualizeDataModule} для визуализации информационных образов.

\subsection{Руководство по использованию программного модуля}
\label{sub:manual:us}

Для начала работы с модулем необходимо создать объекты подмодулей с необходимыми параметрами. 

Для считывания музыкального произведения необходимо создать объект класса \texttt{WavModule}, который не принимает параметров.

\begin{lstlisting}[language=TypeScript, label=lst:testing:results]
	wav_module = WavModule()
\end{lstlisting}

Для создание модуля препроцессинга и нарезки необходимо создать объект класса \texttt{PreprocessingModule}, где в конструкторе указывать следующие параметры:
\begin{itemize} 
\item \texttt{alpha} -- коэффициент экспоненициального сглаживания;
\item \texttt{cut\_end} -- процент трека, которое будет отсечено с конца трека;
\item \texttt{cut\_start} -- процент трека, которое будет отсечено с конца трека;
\item \texttt{overlap} -- процент пересечения фрагментов трека;
\item \texttt{ frame\_size\_sec} -- длина каждого фрагмента в секундах.
\end{itemize}

\begin{lstlisting}[language=TypeScript, label=lst:testing:results]
	preprocessing_module = PreprocessingModule(
					  alpha=0.99,
                                           cut_end=0.2,
                                           cut_start=0.2,
                                           overlap=0.1,
                                           frame_size_sec=5)
\end{lstlisting}

Для создания модуля получения частотно-временного представления сигнала необходимо создать объект класса \texttt{SpectralTransformer}, где в конструкторе указывать слежующие параметры:

\begin{itemize} 
\item \texttt{alpha} -- коэффициент экспоненициального сглаживания;
\item \texttt{level}  -- количество каскадов при вейвлет преобразовании Добеши;
\item \texttt{rate}  -- процент пересечения фрагментов трека;
\item \texttt{window} -- массив содержащий окно.
\end{itemize}

\begin{lstlisting}[language=TypeScript, label=lst:testing:results]
       window = signal.hamming(4096)
       spectral_transformer = SpectralTransformer(
       					  alpha=0.99,
                                           level=4,
                                           rate=16,
                                           window=window)
\end{lstlisting}

Для создания модуля извленчения информационных образов необходим необходимо создать объект класса  \texttt{FeatureProcessing}, где в конструкторе указыватеются следующие параметры:

\begin{itemize} 
\item \texttt{models} -- ассоциативный массив, где ключ является наследником класса \texttt{FeatureExtractorModel}, а значение - массив зависимых признаков ;
\item \texttt{nceps}  -- количество мел-кепстральных коэффициентов;
\end{itemize}

\begin{lstlisting}[language=TypeScript, label=lst:testing:results]
       models = {
		Energy: [],
    		SpectralSmoothness: [],
     		SpectralSpread: [SpectralCentroid],
       		...
       }
       feature_extractor = FeatureExtractor(
       						   models=models,
       						    nceps=24)
\end{lstlisting}

Для создания модуля обработки информационных образов необходимо создать объект класса  \texttt{FeatureProcessing}, где в конструкторе указыватеются следующие параметры:

\begin{itemize} 
\item \texttt{with\_mean} -- добавление в информационный вектор МО спектральных признаков и мел-кепстральных коэффициентов;
\item \texttt{with\_std}  -- добавление в информационный вектор СКО спектральных признаков и мел-кепстральных коэффициентов;
\item \texttt{with\_kurtosis}  -- добавление в информационный вектор коэффициента эксесса спектральных признаков и мел-кепстральных коэффициентов;
\item \texttt{with\_skew} -- добавление в информационный вектор коэффициента склонения спектральных признаков и мел-кепстральных коэффициентов;
\end{itemize}

\begin{lstlisting}[language=TypeScript, label=lst:testing:results]
        feature_processing=FeatureProcessing(
        				with_mean=True,
        				with_std=True
        				with_kurtosis=False,
                                           with_skew=True)
\end{lstlisting}



Далее для созданные объекты отправляются в конструктор класса \texttt{MainModule}.


\begin{lstlisting}[language=TypeScript, label=lst:testing:results]
	main_module = MainModule( 
				wav_module,
				preprocessing_module,
				spectral_transformer,
				feature_extractor,
				feature_processing
				 
\end{lstlisting}

Объект класс сконфигурирован для извлечения информационных образов. Далее для получения образа вызывается  метод \texttt{get\_feature}, которому необходимо передать путь к MP3 или WAV файлу, а также метаинформациию по необходимости. Метод вернёт массив объектов класса \texttt{TrackModel}. И для получения вектора информационных образов необходимо вызвать метод \texttt{to\_vector()}. Если же информационные образы необходимо сохранить в базе данных MongoDB, то создаётся модуль хранения информационных образов.

Для создания модуля для хранения информациионных образов необходимо создать объект класса \texttt{DatabaseModule},  где в конструкторе указывается ip адрес и порт сервера MongoDB.

\begin{lstlisting}[language=TypeScript, label=lst:testing:results]
	db = DatabaseModule('localhost', 27017)
\end{lstlisting}

Сохранения информационного образа  происходит с помощью вызова  метода \texttt{store}.


Для использования модуля жанровой классификациии необходимо создать объект класса \texttt{GenreClassificationModule}, где в конструкторе  указывается следующие параметры:

\begin{itemize} 
\item \texttt{labels\_name} -- массив строк содержащий название жанров;
\item \texttt{cv}  -- количество разбиений при перекрёстной проверки;
\item \texttt{classifiers}  -- ассоциативный массив методов классификации, где ключом является название метода, а значением  -- объект класса;
\end{itemize}

\begin{lstlisting}[language=TypeScript, label=lst:testing:results]
clf = {
    'Nearest Neighbors 3': KNeighborsClassifier(3),
    'Decision Tree': DecisionTreeClassifier(),
    'AdaBoost': AdaBoostClassifier(),
    'Gaussian Naive Bayes': GaussianNB(),
    'QDA': QuadraticDiscriminantAnalysis(),
    ...
}

clf_module = GenreClassificationModule(
                  cv=10, 
                  labels_name=genre_list, 
                  classifiers=clf)
				 
\end{lstlisting}

Затем вызвать метод \texttt{classify}, который вернёт ассоциативный массив, где ключ -- это название метода классификации, а значение массив содержащий МО оценки перекрёстной проверки, СКО этой оценки и матрицу ошибок классификации. Метод \texttt{classify} принимает первым значением массив информационных образов, вторым -- их кассы и метаинформацию о образе. Метаинформации используется при классификации всего трека, а не отдельных фрагметов. Перед началом выполнения классификации рекомендуется нормализовать данные по каждому признаку.

\begin{lstlisting}[language=TypeScript, label=lst:testing:results]
X, Y, meta = load_data()

result = module.classify(X, Y, meta)
\end{lstlisting}

Для создания модуля визуализации необходимо создать объект класса \texttt{VisualizeDataModule}, который имеет два метода: \texttt{plot\_2d} -- для отображение в двумерную плоскость и \texttt{plot\_3d} -- для отображения пространстав признаков в трёхмерное пространство. И меют одинаковую сигнатуру. Первым аргументом идёт массив информационных образов, вторым -- их классы, флаг отоборажения в отдельном окне и метод уменьшения размерности. Класс поддерживает два метода уменьшения размерности: t-SNE для нелинейного распределения и PCA (метод главных компонент) для линейного.

\begin{lstlisting}[language=TypeScript, label=lst:testing:results]
visualizer = VisualizeDataModule()

visualizer.plot_2d(X, labels=Y, 
                   genre_list=genre_list, 
                   reduction_method='PCA')
visualizer.plot_3d(X, labels=Y, 
                   genre_list=genre_list, 
                   reduction_method='t-SNE')

\end{lstlisting}