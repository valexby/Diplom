\sectioncentered*{ВВЕДЕНИЕ}
\addcontentsline{toc}{section}{ВВЕДЕНИЕ}
\label{sec:intro}

На сегодняшний день количество музыкальный треков в Интернете не поддаётся подсчёту. В одном только сервисе Яндекс.Музыка доступно 17 миллионов треков  \cite{yandex_blog} . Сегодня выпускается тысячи песен каждый день \cite{forbes}. Существует сотни жанров и поджанров музыки. Поэтому задача поиска новой музыки становится нетривиальной. 

Для решение данной задачи используются рекомендательные сервисы, которые используют четыре подхода к анализу музыки для составления рекомендации:
\begin{enumerate}[label=\arabic*.]
\item Популярность трека. Использование данных о популярности трека и информации о пользователях. Этот способ доступен только большим онлайн сервисам. Качество рекомендации зависит от количества пользователей данного сервиса. Популярность треков определяется количеством и качеством отзывов и количеством покупок. 
\item Метаинформация о треке. Это способ также доступен большим онлайн сервисам с большой библиотекой музыки. Качество рекомендации зависит от размеров библиотеки. Метаинформация о треке представляет собой тег с жанром, информация об исполнителе, альбом в который включён этот трек и т.д.
\itemСемантика текста. Анализируя музыкальные блоги с применением технологий обработки естественных языков подход позволяет постоянно изучать веб и аналитически просматривать десятки миллионов страниц, имеющих отношения к музыке. Текст песни анализируется на основе взвешенных лексем. Произведения рекомендуется при близости его дескрипторов с дескрипторами пользователя.
\itemАкустико-синтаксический анализ. На текущий момент нет методов достоверного распознавания музыкальных инструментов и музыкальных звуков. Однако, несмотря на это анализ сигнала играет очень важную роль используется в работе рекомендательных алгоритмов. Анализируются фрагменты произведения длительности от 200 мс до 4 с, в зависимости от вариабельности мелодии.  Для каждого сегментов произведения определяется громкость, тембр,  выявляются наиболее громко звучащие музыкальные инструменты; устанавливается к какой части композиции (припев, куплет и т. д.) относится этот сегмент.
\end{enumerate}

В первом способе информационные признаки о треке получается на основе анализа поведения пользователя. Рекомендации при использовании данного метода делаются на основе коллаборативной фильтрации на основе соседей. Во втором методе является метаинформация о треке, а рекомендация делается на основе  коллаборативной фильтрации на основе модели. В современных мультимедийных сервисах(youtube, lastfm, Яндекс.Музыка)  используется гибридный подход, которых объединяет в себе подход основанный на соседстве и основанный на модели \cite{yandex_blog_1} \cite{ibm_blog}.

Целью данного дипломного проекта является создание  модуля, который выделял информационные образы из музыкального трека только на основании  акустического анализа. Информационный образ - это признаковое описание трека как музыкального сигнала. Такой сервис должен быть лишён субъективности и обеспечить качество рекомендации.

В соответствии с поставленной целью были определены следующие задачи:

\begin{itemize}
\item выделение спектральных, временных и иных признаков из музыкального трека;
\item проверка значимости признаков путём использовании их в задаче жанровой классификации(задача настройки классификатора решаться не будет);
\item визуализация данных алгоритмом t-SNE.
\end{itemize}
