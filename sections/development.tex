\section{РАЗРАБОТКА ПРОГРАММНЫХ МОДУЛЕЙ}
\label{sec:dev}

При разработке системы одними из наиважнейших требований к исходному коду являются его расширяемость и поддерживаемость. Реализация программных модулей с учетом этих требований приводит к простоте расширения функционала в критических местах, обеспечению разделенности и независимости компонентов системы, что улучшает их тестируемость и в целом позволяет добиться реализации более стабильной и простой в понимании кодовой базы.

\subsection{Вычисления временных и спектральных признаков}

В классе \texttt{FeatureExtractor} для упрощения добавления способов получения признаков, которые могут иметь зависимости между собой, используется абстрактный класс \texttt{FeatureExtractorModel} для декларации интерфейса и два абстрактных класса для различения признаков по временным и спектральным - \texttt{TimingFeature} и \texttt{SpectralFeature} соответственно. В сигнатуре  метода \texttt{get(data, params)}  \texttt{params} -- это параметр, через который передаётся вычисленный признак и который необходим для вычисления текущего.
 
\begin{lstlisting}[language=TypeScript, label=lst:abstract]

class FeatureExtractorModel:
    __metaclass__ = ABCMeta

    @abstractmethod
    def get(self, data, params=None):
        pass

    def check(self, data):
        if not isinstance(data, np.ndarray):
            raise TypeError("input is not array")

    def normalize(self, result, data):
        return result / float(len(data))
       
        
\end{lstlisting}

Так как вычисления одних признаков зависит от вычисления других, то задача получения всех признаков становится задачей выполнения графа. Поэтому для описание графа используется ассоциативный массив, где ключ -- тип метода выделения признака, который наследуется либо от \texttt{TimingFeature}, либо от  \texttt{SpectralFeature}, а значение массив таких же типов результаты которых передаётся типу ключа.

\begin{lstlisting}[language=TypeScript, label=lst:graph]

models = {
    Energy: [],
    ZeroCrossingRate: [],
    Autocorrelation: [],
    SpectralCentroid: [],
    SpectralSmoothness: [],
    SpectralSpread: [SpectralCentroid],
    SpectralDissymmetry: [SpectralCentroid],
    Rolloff: [],
    LinearRegression: [],
    SFM: [],
    SCF: []
}
        ...
\end{lstlisting}

Как видно из примера, классам \texttt{SpectralSpread} и  \texttt{SpectralDis-} \texttt{symmetry} требуется результат вычисления класса \texttt{SpectralCentroid}. Для получения всех признаков  класс \texttt{FeatureExtractor} интеративно вычисляет признаки, которые не зависят от результат вычисления других признаков, а потом те, которые зависят только от одного и так далее. Значение признаков заносится в ассоциативный массив, где ключ -- это тип, а значение -- это результат вычисление данного признака.

\begin{lstlisting}[language=TypeScript, label=lst:evalgraph]

    def eval_models(self, extractors, data):
        max_length = np.amax(map(lambda x: len(x), extractors.values())) + 1)
        for i in range(max_length):
            for feature_extractor in filter(lambda x: len(extractors[x]) == i, extractors):
                self.results[feature_extractor] = \
                    feature_extractor() \
                        .get(data,
                                map(lambda x: self.results[x] if x in self.results else None,
                                 extractors[feature_extractor]))
        ...
        
\end{lstlisting}

\subsection{Параллельное распараллеливание вычислений в модуле \\ \texttt{GenreClassificationModule}}

В модуле классификации \texttt{GenreClassificationModule} для ускорения классификации было использованно распараллеливание на несколько процессов. Для это использовалась библиотека joblib, которая позволяет выполнять циклы на всех ядрах процессора. Главным ограничением это библиотеки -- распараллелить можно только функцию вне класс. Поэтому вся логика классификации была вынесена в отдельную функцию. 

\begin{lstlisting}[language=TypeScript, label=lst:classify]

   def classify(self, data, labels, meta):
        temp = Parallel(n_jobs=CPU_COUNT)(
            delayed(classify_p)
            (self.classifiers[name], name, data, labels, self.cv, meta) for name in self.classifiers
        )
        result = dict()
        for i in temp:
            result.update(i)
        return result



\end{lstlisting}

Результатом работы функции \texttt{classify\_p} является ассоциативный массив, где ключ -- название метода классификации, а значение представляет собой массив, который содержит в себе:
\begin{itemize}
    \item МО оценки перекрёстной проверки;
    \item CКО оценки перекрёстной проверки;
    \item матрицу ошибок классификации.
\end{itemize}

Так как функция \texttt{Parallel} возвращает массив результатов функции, которую распараллелили, то необходимо собрать массив одноэлементных словарей в одни большой массив.


\subsection{Отображение матрицы ошибок классификации}

В классе \texttt{GenreClassificationModule} для  отображения матрицы ошибок классификации матрицы используется библиотека Matplotlib, которая позволяет отображать многие виды графиков и диаграмм. Для большей наглядности цвет фона ячейки матрицы зависит от значения, которое находится внтури. В качестве цвета фона используется синий цвет в градации от белого до почти чёрного.

\begin{lstlisting}[language=TypeScript, label=lst:normalize]
    plt.imshow(cm, interpolation='nearest', cmap=cmap)
    plt.colorbar()
\end{lstlisting}

На вход функции отображения подаётся матрица ошибок с количеством распознаных образом. Для нормализации данных для каждого ряда считается сумма и значение каждого элемента ряда делится на сумму, умножается на 100 и округляется.

\begin{lstlisting}[language=TypeScript, label=lst:normalize]
    cm = np.round(cm.astype('float') /
         cm.sum(axis=1)[:, np.newaxis] * 100)\
         .astype('int')

\end{lstlisting}

Определяется порог при котором цвет шрифта внутри клетки меняется с чёрного на белый.

\begin{lstlisting}[language=TypeScript, label=lst:normalize]

    thresh = cm.max() / 2.
    for i, j in itertools.product(range(cm.shape[0]), range(cm.shape[1])):
        plt.text(j, i, cm[i, j],
            horizontalalignment="center",
            color="white" if cm[i, j] > thresh else "black")

\end{lstlisting}

