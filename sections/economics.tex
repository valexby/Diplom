\newcommand{\byr}{ руб.}

\section{ЭКОНОМИЧЕСКОЕ ОБОСНОВАНИЕ РАЗРАБОТКИ  МОДУЛЯ \\ ВЫДЕЛЕНИЯ ИНФОРМАЦИОННЫХ ОБРАЗОВ ИЗ МУЗЫКАЛЬНОГО ПРОИЗВЕДЕНИЯ}
\label{sec:econ}

\subsection{Описание функций, назначения и потенциальных пользователей ПО}
\label{sub:econ:overview_appointment}
В рамках дипломного проекта был разработан модуль выделения информационных признаков из музыкального произведения.

Программный модуль решает задачи выделение спектральных, временных и иных признаков из музыкального трека, проверка значимости признаков путём использовании их в задаче жанровой классификации, визуализация данных алгоритмом t-SNE.

Целью данного дипломного проекта является создание  модуля, который выделял информационные образы из музыкального трека только на основании  акустического анализа

Разрабатываемый программный продукт относится к категории программного обеспечения, разрабатываемого по индивидуальному заказу для использования внутри организации-заказчика.

Актуальность разработки данного ПО объясняется тем, существует потребность в наличии доступного инструмента, позволяющего выделять информационные образы из музыкального трека только на основании  акустического анализа.

Целью экономического обоснования программных средств является определение экономической выгоды создания данного продукта и дальнейшего его применения.

\subsection{Расчёт затрат на разработку ПО}
\label{sub:econ:expenses}


Данный модуль может быть использован в сервисе рекомендации, который позволит значительно улучшить качество рекомендаций. Также потенциальным пользователем, может быть любой человек, который хочет анализировать данные в музыкальной предметной области. Разработка программного средства прозводится по индивидуальному заказу организации-заказчика, однако в будущей перспективе может выйти для свободной реализации на рынке.

Для оценки экономической эффективности разработанного программного обеспечения проводится расчет затрат на разработку приложения, оценка прибыли от продажи одного такого приложения и расчет показателей эффективности инвестиций в его разработку.

Разрабатываемый программный продукт относится ко второй категории сложности, так как существует возможность переносимости программного обеспечения.

По степени новизны разрабатываемая система автоматизации относится к категории В с коэффициентом новизны Кн = 0,7, так как было выявлено несколько аналогов. Но стоит принять во внимание тот факт, что данный проект не подразумевает в себе использование принципиально нового типа электронно-вычислительных машин, операционных систем.

При разработке проекта используются существующие технологии и средства разработки, которые охватывают около 20 – 30\% реализуемых функций, поэтому коэффициент использования стандартных модулей принимается равным 0,8.

При расчете сметы затрат будут использоваться данные, приведенные в таблице \ref{table:econ:profit}.

\begin{table}[!ht]
\caption{Исходные данные для расчета}
\label{table:econ:profit}
  \centering
  \begin{tabular}{| >{\raggedright}m{0.45\textwidth}
                  | >{\centering}m{0.15\textwidth}
                  | >{\centering}m{0.15\textwidth}
                  | >{\centering\arraybackslash}m{0.15\textwidth}|}
    \hline
    {\begin{center}
      Наименования показателей
    \end{center} } & Буквенные обозначения & Ед.измере-ния & Количество \\
    \hline
    \begin{center}
      1
    \end{center} & 2 & 3 & 4 \\

    \hline
    Фонд социальной защиты населения (от заработной платы) & $ \text{Н}_{\text{соц}} $ & \% & 34 \\

    \hline
    Обязательное страхование (от несчастных случаев на производстве, от заработной платы) & $ \text{Н}_{\text{стр}} $ & \% & 0,6 \\

    \hline
    Налог на прибыль & $ \text{Н}_{\text{приб}} $ & \% & 18 \\

    \hline
    Налог на недвижимость (от стоимости зданий и сооружений) & $ \text{Н}_{\text{недв}} $ & \% & 1 \\

    \hline
    НДС (Налог на добавленную стоимость) & $ \text{НДС} $ & \% & 20 \\

    \hline
    Норма дисконта & $ \text{Е}_{\text{н}} $ & \% & 17 \\

    \hline
    Тарифная ставка 1-го разряда & $ \text{Т}_{\text{м1}} $ & \byr & 31 \\

    \hline
    Часовая тарифная ставка 1-го разряда & $ \text{T}_{\text{ч}} $ & \byr & 0,19 \\

    \hline
    Установленный фонд рабочего времени & $ \text{Ф}_{\text{рв}} $ & часов & 166 \\

    \hline
    Продолжительность рабочего   дня & $ \text{T}_{\text{ч}} $ & часов & 8 \\

    \hline
    Тарифный коэффициент & $ \text{T}_{\text{к}} $ & - & 2,03 \\

    \hline
    Коэффициент премирования & $ \text{К}_{\text{п}} $ & единиц & 1,5 \\

    \hline

  \end{tabular}
\end{table}

Основой для расчёта сметы затрат является основная заработная плата разработчиков проекта. Затраты на основную заработную плату команды разработчиков определяются исходя из состава и численности команды, размеров месячной заработной платы каждого из участников команды, а также общей трудоемкости разработки программного обеспечения.

Для осуществления упрощённого расчёта затрат на разработку ПО следует произвести расчёт следующих статей:
\begin{itemize}
\item затраты на основную заработную плату разработчиков;
\item затраты на дополнительную заработную плату разработчиков;
\item отчисления на социальные нужды;
\item прочие затраты (амортизация оборудования, расходы на электроэнергию, командировочные расходы, накладные расходы и т.п.).
\end{itemize}

Расчёт затрат на основную заработную плату разработчиков осуществляется на основе численности и состава команды, размеров месячной заработной платы каждого из участников команды, а также общей трудоёмкости процесса разработки программного обеспечения.

В данном случае имеются четыре работника – два программиста I-й категории, руководитель проекта и специалист по анализу данных.
Месячная тарифная ставка каждого исполнителя определяется по формуле:
\begin{equation}
  \label{eq:econ:stavka}
  \text{T}_{\text{м}} = \text{Т}_{\text{м1}} \cdot
                        \text{Т}_{\text{к}}
                        \text{\,,}
\end{equation}
\begin{explanation}
где & $ \text{Т}_{\text{м1}} $ & месячная тарифная ставка первого разряда, \byr; \\
    & $ \text{Т}_{\text{к}} $ & тарифный коэффициент.
\end{explanation}

Месячная тарифная ставка руководителя проекта составит:
\begin{equation}
  \label{eq:econ:stavka_calc}
  \text{T}_{\text{м.рук}} = \num{19,35} \times \num{31} = \num{600} {\text{\byr}}
\end{equation}

Месячная тарифная ставка программиста составит:

\begin{equation}
  \label{eq:econ:stavkap_calc}
  \text{T}_{\text{м.прог}} = \num{16,13} \times \num{31} = \num{500} {\text{\byr}}
\end{equation}

Месячная тарифная ставка специалиста по анализу данных составит:

\begin{equation}
  \label{eq:econ:stavkap_calc}
  \text{T}_{\text{м.спец}} = \num{19,35} \times \num{31} = \num{600} {\text{\byr}}
\end{equation}


Исходя из месячной тарифной ставки рассчитывается часовая тарифная ставка:
\begin{equation}
  \label{eq:econ:hour_stav}
  \text{Т}_{\text{ч}} =
    \frac{\text{Т}_{\text{м}}}
         {\text{Ф}_{\text{р}}} \text{\,,}
\end{equation}

Часовая тарифная ставка руководителя проекта в соответствии с формулой составит:
\begin{equation}
  \label{eq:econ:hourly_rate}
  \text{Т}_{\text{ч.рук}} =
    \frac{\num{600}}
         {\num{166}} = \num{3,61}{\text{\byr}}
\end{equation}

Часовая тарифная ставка программиста первой категории в соответствии с формулой составит:

\begin{equation}
  \label{eq:econ:hourly_rate_p}
  \text{Т}_{\text{ч.прог}} =
    \frac{\num{500}}
         {\num{166}} = \num{3,01}{\text{\byr}}
\end{equation}

Часовая тарифная ставка специалиста по анализу данных  в соответствии с формулой составит:

\begin{equation}
  \label{eq:econ:hourly_rate_p}
  \text{Т}_{\text{ч.спец}} =
    \frac{\num{600}}
         {\num{166}} = \num{3,61}{\text{\byr}}
\end{equation}


Основная заработная плата исполнителей на конкретное программное средство является суммой заработных плат каждого из исполнителей в отдельности и определяется по формуле:

\begin{equation}
  \label{eq:econ:total_salary}
  \text{З}_{\text{о}} = \sum^{n}_{i = 1}
                        \text{Т}_{\text{чi}} \cdot
                        \text{t}_{\text{i}}
                        \text{\,,}
\end{equation}
\begin{explanation}
где & $ \text{n} $ & количество исполнителей, занятых разработкой конкретного ПО; \\
    & $ \text{Т}_{\text{чi}} $ & часовая тарифная ставка \mbox{$ i $-го} исполнителя, руб; \\
    & $ \text{t}_{\text{i}} $ & трудоемкость работ, выполняемых \mbox{$ i $-го} исполнителем, час.
\end{explanation}

Для руководителя, при заработной плате равной 600 рублей, часовая заработная плата равна 3,61 рубля. Для разработчика, при заработной плате равно 500 рублей, часовая заработная плата равна 3,01 рубля. Для специалиста по анализу данных, часовая заработная плата равна 3,61.
Трудоемкость определяется исходя из сложности разработки программного продукта и объема выполняемых им функций. В нашем случае она составляет 90 дней или 720 часов.
Тогда основная зарплата исполнителей равна:

\begin{equation}
  \label{eq:econ:total_salary_calc}
  \text{З}_{\text{о}} = (\num{3,01} + \num{3,61} + \num{3,01} + \num{3,61}) \times \num{720} = \num{9532,8} {\text{\byr}}
\end{equation}

\begin{table}[!ht]
\caption{Расчет затрат на основную заработную плату команды}
\label{table:econ:initial_data}
  \centering
  \begin{tabular}{| >{\raggedright}m{0.02\textwidth}
                  | >{\centering}m{0.17\textwidth}
                  | >{\centering}m{0.16\textwidth}
                  | >{\centering}m{0.12\textwidth}
                  | >{\centering}m{0.11\textwidth}
                  | >{\centering}m{0.11\textwidth}
                  | >{\centering\arraybackslash}m{0.12\textwidth}|}
    \hline
    {\begin{center}
    №
    \end{center} } & Участник команды & Выполня-емые работы & Месячная заработная плата, р & Часовая заработная плата, р. & Трудоем-кость работ, ч. & Основная заработная плата, р. \\
    \hline
    1 & Руководитель проекта & Контроль, помощь & \num{600} & \num{3,61} & \num{720} & \num{2599,2} \\

    \hline
    2 & Программист 1-й категории & Разработка & \num{500} & \num{3,01} & \num{720} & \num{2167,2} \\

    \hline
    3 & Программист 1-й категории & Разработка & \num{500} & \num{3,01} & \num{720} & \num{2167,2} \\

    \hline
    4 & Специалиста по анализу данных & Разработка & \num{600} & \num{3,61} & \num{720} & \num{2599,2} \\

    \hline
    \multicolumn{6}{|c|}{ПРЕМИЯ (50\%)} & \num{4766,4} \\

    \hline

    \multicolumn{6}{|c|}{Итого затраты на основную заработную плату разработчиков} & \num{14299,2}\\

    \hline

  \end{tabular}
\end{table}

Затраты на дополнительную заработную плату команды разработчиков включает выплаты, предусмотренные законодательством о труде (оплата отпусков, льготных часов, времени выполнения государственных обязанностей и других выплат, не связанных с основной деятельностью исполнителей), и определяется по формуле:
\begin{equation}
  \label{eq:econ:additional_salary}
  \text{З}_{\text{д}} =
    \frac {\text{З}_{\text{о}} \cdot \text{Н}_{\text{д}}}
          {\num{100}} \text{\,,}
\end{equation}
\begin{explanation}
    где & $ \text{З}_{\text{о}} $ & затраты на основную заработную плату с учетом премии, руб; \\
        & $ \text{Н}_{\text{д}} $ & норматив дополнительной заработной платы, \num{15} \%.
\end{explanation}

В результате подстановки получим:

\begin{equation}
  \label{eq:econ:additional_salary_calc}
  \text{З}_{\text{д}} =
    \frac{\num{14299,2} \times \num{15}}
         {\num{100}} = \num{2144,88}{\text{\byr}}
\end{equation}

Отчисления на социальные нужды включают в предусмотренные законодательством отчисления в фонд социальной защиты (34\%) и фонд обязательного страхования (0,6\%) в процентах от основной и дополнительной заработной платы и вычисляются по формуле:

\begin{equation}
  \label{eq:econ:com}
  \text{З}_{\text{соц}} =
    \frac {(\text{З}_{\text{о}} + \text{З}_{\text{д}} )\cdot \text{Н}_{\text{соц}}}
          {\num{100}} \text{\,,}
\end{equation}
\begin{explanation}
    где & $ \text{Н}_{\text{соц}} $ & норматив отчислений на социальные нужды \num{34}\% и норматив отчислений в фонд социального страхования, \num{0,6} \% .
\end{explanation}

\begin{equation}
  \label{eq:econ:com_calc}
  \text{З}_{\text{соц}} =
    \frac{(\num{14299,2} + \num{2144,88}) \times \num{34,6}}
         {\num{100}} = \num{5689,65}{\text{\byr}}
\end{equation}

Расходы по статье «Прочие затраты» включают затраты на приобретение и подготовку специальной научно-технической информации и специальной литературы. 

Определяются по формуле:

\begin{equation}
  \label{eq:econ:etc}
  \text{З}_{\text{пз}} =
    \frac {\text{З}_{\text{о}} \cdot \text{Н}_{\text{пз}}}
          {\num{100}} \text{\,,}
\end{equation}
\begin{explanation}
    где  & $ \text{Н}_{\text{пз}} $ & норматив прочих затрат в целом по организации, \num{100} \% .
\end{explanation}
\begin{equation}
  \label{eq:econ:etc}
  \text{З}_{\text{пз}} =
    \frac{\num{14299,2} \times \num{100}}
         {\num{100}} = \num{14299,2}{\text{\byr}}
\end{equation}

Общая сумма расходов по смете определяется:
\begin{equation}
  \label{eq:econ:Profit}
  \text{З}_{\text{п}} = \text{З}_{\text{о}} + \text{З}_{\text{д}} + \text{З}_{\text{соц}} + \text{Р}_{\text{пр}} \text{\,,}
\end{equation}
Подставив рассчитанные ранее значения в формулу, получим:

\begin{equation}
  \label{eq:econ:Profit_calc}
  \text{З}_{\text{п}} = \num{14299,2} + \num{2144,88} + \num{5689,65} + \num{14299,2} = \num{36432,93} {\text{\byr}}
\end{equation}

Полная сумма затрат на разработку программного обеспечения находится путем суммирования всех рассчитанных статей затрат (таблица \ref{table:econ:expenses}).

\begin{table}[!ht]
\caption{Затраты на разработку программного обеспечения}
\label{table:econ:expenses}
  \centering
  \begin{tabular}{| >{\raggedright}m{0.75\textwidth}
                  | >{\centering\arraybackslash}m{0.17\textwidth}|}
    \hline
    {\begin{center}
      Статья затрат
    \end{center} } & Сумма,\byr \\
    \hline
    Основная заработная плата команды разработчиков & \num{14299,2}\\

    \hline
    Дополнительная заработная плата команды разработчиков & \num{2144,88} \\

    \hline
    Отчисления на социальные нужды & \num{5689,65}  \\

    \hline
    Прочие затраты & \num{14299,2}  \\

    \hline
    Общая сумма затрат на разработку & \num{36432,93}  \\

    \hline

  \end{tabular}
\end{table}

\subsection{Оценка результата (эффекта) от продажи ПО}
\label{sub:econ:evaluation_result}

Экономический эффект представляет собой прирост чистой прибыли,
полученный организацией в результате использования разработанного ПО. Как правило, он может быть достигнут за счет:

\begin{itemize}
\item уменьшения (экономии) затрат на заработную плату за счет замены «ручных» операций и бизнес-процессов информационной системой;
\item ускорения скорости обслуживания клиентов и рост возможности обслуживания большего их количества в единицу времени, т.е. рост производительности труда;
\item появления нового канала сбыта продукции или получения заказов (как в случае внедрения интернет-магазина);
\item и т.п.
\end{itemize}

Экономический эффект организации-разработчика программного обеспечения в данном случае заключается в получении прибыли от его продажи множеству потребителей. Прибыль от реализации в данном случае напрямую зависит от объемов продаж, цены реализации и затрат на разработку данного ПО.

При свободной реализации на рынке IT экономический эффект заключается в получении прибыли от его продажи множеству потребителей. Прибыль от реализации в данном случае напрямую зависит от объемов продаж, цены реализации и затрат на разработку данного проекта. Организация, осуществляющая финансово-хозяйственную деятельность, заинтересована не только в наибольшей величине прибыли, но и в отдаче вложенных средств. Отдача или эффективность вложенных средств характеризуется размером прибыли, получаемой предприятием. Показатели рентабельности характеризуют эффективность работы организации в целом, а также доходность различных направлений деятельности.

Далее следует определить цену на одну копию (лицензию) ПО. Цена формируется на основе затрат на разработку и реализацию ПО. Тогда расчет прибыли от продажи одной копии (лицензии) ПО осуществляется по формуле:
\begin{equation}
  \label{eq:econ:profit}
   \text{Ц} = \text{П}_{\text{ед}} + \text{НДС} +
    \frac {\text{З}_{\text{р}}}
          {\text{N}} \text{\,,}
\end{equation}
\begin{explanation}
    где & $ \text{Ц} $ & цена реализации одной копии (лицензии) ПО, \byr; \\
        & $ \text{З}_{\text{р}} $ & сумма расходов на разработку и реализацию программного обеспечения, \byr  В дипломном проекте расходы на реализацию приняты равным 5\%; \\
        & $ \text{N} $ & количество копий (лицензий) ПО, которое будет куплено клиентами за год, \byr; \\
        & $ \text{П}_{\text{ед}} $ & прибыль, получаемая организацией-разработчиком от реализации одной копии программного продукта, \byr; \\
        & $ \text{НДС} $ & сумма налога на добавленную стоимость, \byr;
\end{explanation}

Прибыль от продажи одной копии (лицензии) ПО осуществляется по формуле:
\begin{equation}
  \label{eq:econ:one_copy}
  \text{П}_{\text{ед}} =
    \frac{\text{С}_{\text{п}} \cdot \text{У}_{\text{р}}}
         {\num{100} \times  \text{N}} \text{\,,}
\end{equation}
\begin{explanation}
    где & $ \text{С}_{\text{п}} $ & себестоимость ПО, \byr; \\
        & $ \text{У}_{\text{р}} $ & запланированный уровень рентабельности, \num{35} \%.
\end{explanation}

\begin{equation}
  \label{eq:econ:one_copy_calc}
  \text{П}_{\text{ед}} =
    \frac{\num{36432,93} \times \num{35}}
         {\num{100} \times \num{20}} = \num{637,6}{\text{\byr}}
\end{equation}

Цена одной копии (лицензии) программного обеспечения данного направления составляет 637,6 руб.

Следовательно годовая прибыль составит:

\begin{equation}
  \label{eq:econ:prib}
  \text{П} = \num{637,6} \times \num{20} = \num{12751,5} {\text{\byr}}
\end{equation}

Подставив рассчитанные ранее значения в формулу, получим:

\begin{equation}
  \label{eq:econ:prib_calc}
  \text{Ц} = \num{637,6} +
   \frac {\num{36432,93} \times \num{1,05}}
         {\num{20}} = \num{2550,32} \text{\byr}
\end{equation}

Рентабельность затрат рассчитаем по формуле. Проект будет экономически эффективным, если рентабельность затрат на разработку программного средства будет не меньше средней процентной ставки по банковским депозитным вкладам.
\begin{equation}
  \label{eq:econ:profitability}
  \text{Р} = \num{100} \cdot
    \frac {\text{П}}
          {\text{З}_{\text{р}}} \text{\,,}
\end{equation}
\begin{explanation}
    где & $ \text{П} $ & годовая прибыль, \byr; \\
        & $ \text{З}_{\text{р}} $ & сумма расходов на разработку и реализацию, \byr
\end{explanation}

\begin{equation}
  \label{eq:econ:profitability}
  \text{Р} = \num{100} \times
    \frac {\num{12751,5}}
          {\num{36432,93}} = 35\%.
\end{equation}

Средняя процентная ставка по депозиту за март 2017 года составляет 7,9\% для юридических лиц, что меньше рентабельности данного проекта. Соответственно проект является экономически эффективным.
Учитывая налог на прибыль, можно рассчитать итоговую сумму, которая останется разработчику и будет является его экономическим эффектом:
\begin{equation}
  \label{eq:econ:net_profit}
  \text{ЧП} = \text{П} -
    \frac {\text{П} \cdot \text{Н}_{\text{приб}}}
          {\num{100}} \text{\,,}
\end{equation}
\begin{explanation}
    где & $ \text{Н}_{\text{приб}} $ & ставка налога на прибыль, \num{18} \% ; \\
    & $ \text{П} $ & прогнозируемая прибыль, \byr.
\end{explanation}

\begin{equation}
  \label{eq:econ:net_profit_calc}
  \text{ЧП} = \num{12751,5} -
    \frac {\num{12751,5} \cdot \num{18}}
          {\num{100}} = \num{10456,23} \text{\byr}
\end{equation}

Чистая прибыль от реализации ПО (ЧП = 10456,23 рублей) остается организации-разработчику и представляет собой экономический эффект.

\subsection{Расчёт показателей эффективности инвестиций в разработку ПО}
\label{sub:econ:investment_efficiency}

Так как сумма инвестиций в разработку ПО больше суммы годового экономического эффекта, то экономическая целесообразность инвестиций в разработку и использование программного средства осуществляется на основе расчёта и оценки следующих показателей:

\begin{itemize}
\item чистый дисконтированный доход (ЧДД);
\item срок окупаемости инвестиций;
\item рентабельность инвестиций.
\end{itemize}

Метод чистой дисконтированной доходности основан на сопоставлении дисконтированной стоимости денежных поступлений (инвестиций), генерируемых предприятием в течение прогнозируемого периода. Целью данного метода является выявление реального размера прибыли, который может быть получен организацией вследствие реализации данного инвестиционного проекта.

Коэффициент дисконтирования соответствующего года t определяется по формуле:
\begin{equation}
  \label{eq:econ:cof}
  \text{a}_{\text{t}} =
    \frac {\text{1} }
          {(\num{1} + \text{E}_{\text{n}})^t} \text{\,,}
\end{equation}
\begin{explanation}
    где & $ \text{E}_{\text{n}} $ & норма дисконта, равная или больше средней процентной ставки по банковским депозитам, действующей на момент проведения расчётов; \\
    & $ \text{t} $ & порядковый номер года периода реализации инвестиционного проекта (1 – 2017, 2 – 2018, 3 – 2019, 4 – 2020).
\end{explanation}

Подставляя значения, получим значения коэффициентов дисконтирования для 2017 – 2020 годов:

\begin{equation}
  \label{eq:econ:cof}
  \text{a}_{\text{1}} =
    \frac {\text{1} }
          {(\num{1} + \num{0,17})^1} = \num{0,85} \text{\,,}
\end{equation}


\begin{equation}
  \label{eq:econ:cof}
  \text{a}_{\text{2}} =
    \frac {\text{1} }
          {(\num{1} + \num{0,17})^2} = \num{0,73} \text{\,,}
\end{equation}


\begin{equation}
  \label{eq:econ:cof}
  \text{a}_{\text{3}} =
    \frac {\text{1} }
          {(\num{1} + \num{0,17})^3} = \num{0,62} \text{\,,}
\end{equation}


\begin{equation}
  \label{eq:econ:cof}
  \text{a}_{\text{4}} =
    \frac {\text{1} }
          {(\num{1} + \num{0,17})^4} = \num{0,53} \text{\,.}
\end{equation}


Чистый дисконтированный доход рассчитывается по следующей формуле:
\begin{equation}
  \label{eq:econ:dohod}
  \text{ЧДД} = \sum^{n}_{t = 1}
                        (\text{P}_{\text{t}} \cdot
                        \text{a}_{\text{t}} - \text{З}_{\text{t}} \cdot
                        \text{a}_{\text{t}})
                        \text{\,,}
\end{equation}
\begin{explanation}
где & $ \text{n} $ & расчётный период, лет; \\
    & $ \text{Р}_{\text{t}} $ & результат (экономический эффект), полученный в году t, руб.; \\
    & $ \text{З}_{\text{t}} $ & затраты (инвестиции в разработку ПО) в году t, руб.
\end{explanation}
\begin{equation}
  \label{eq:econ:dohod}
  \text{ЧДД} = -\num{20129,21} + \num{9308,59} + \num{7905,93} + \num{6758,29} = \num{3843,6} \text{\byr}
\end{equation}

\begin{table}[!ht]
\caption{Расчет эффективности инвестиционного проекта}
\label{table:econ:effect}
  \centering
  \begin{tabular}{| >{\raggedright}m{0.35\textwidth}
                  | >{\centering}m{0.15\textwidth}
                  | >{\centering}m{0.12\textwidth}
                  | >{\centering}m{0.12\textwidth}
                  | >{\centering\arraybackslash}m{0.12\textwidth}|}
    \hline
    {\begin{center}
    Показатели
    \end{center} } & 2017 & 2018 & 2019 & 2020 \\
    \hline
    \multicolumn{5}{|c|}{РЕЗУЛЬТАТ} \\

    \hline
    Экономический эффект & \num{12751,5} & \num{12751,5} & \num{12751,5} & \num{12751,5} \\

    \hline
    Дисконтированный результат & \num{10838,77} & \num{9308,59} & \num{7905,93} & \num{6758,29} \\

    \hline
    \multicolumn{5}{|c|}{\raggedright{ЗАТРАТЫ}} \\

    \hline

    Инвестиции в разработку программного средства & \num{36432,93} & \num{0} & \num{0} & \num{0}\\

    \hline

    Дисконтированные инвестиции & \num{30967,99} & \num{0} & \num{0} & \num{0}\\

    \hline

    Чистый дисконтированный доход по годам & \num{-20129,21} & \num{9308,59} & \num{7905,93} & \num{6758,29}\\

    \hline

    Дисконтированный результат & \num{-20129,21} & \num{-10819,6} & \num{-2913,69} & \num{3843,6}\\

    \hline

    Коэффициент дисконтирования & \num{0,85} & \num{0,73} & \num{0,62} & \num{0,53}\\

    \hline

  \end{tabular}
\end{table}

Рентабельность инвестиций рассчитывается по формуле:

\begin{equation}
  \label{eq:econ:dohod}
  \text{Р}_{\text{и}} =   \frac{\sum^{n}_{t = 0}
                        \text{P}_{\text{t}} \cdot
                        \text{a}_{\text{t}}  }
                        {\sum^{n}_{t = 0}
                        \text{З}_{\text{t}} \cdot
                        \text{a}_{\text{t}}} \cdot \num{100\%}
                        \text{\,,}
\end{equation}

По формуле рентабельность инвестиций равна:

\begin{equation}
  \label{eq:econ:dohod}
  \text{Р}_{\text{и}} =   \frac
  {\num{10838,77} + \num{9308,59} + \num{7905,93} +\num{6758,29} }
                        {\num{30967,99}} \cdot \num{100} = \num{112,3\%}.
\end{equation}

На основании проведенной анализа экономической эффективности разработки системы можно сделать следующие выводы. Из оценки затрат на создание и поддержание программного продукта и прибыли, полученной за продажу копии программы, мы подсчитали рентабельность затрат. Она оказалась выше cредней процентной ставки по депозиту, что показывает выгоду от реализации данной продукции. Показатели эффективности инвестиций на создание программного обеспечения в результате подсчетов показали, что разработка данного программного продукта является экономически целесообразной.