\section{СИСТЕМНОЕ ПРОЕКТИРОВАНИЕ}
\label{sec:sys}


Изучив теоретические аспекты разрабатываемого модуля и выработав список требований необходимых для разработки модуля, разбиваем систему на компоненты. Компоненты в виде блоков и их взаимосвязи указаны на чертеже.
В разрабатываемом модуле можно выделить следующие блоки:
\begin{itemize}
\item модуль чтения музыкального произведения;
\item модуль препроцессинга и нарезки музыкального трека на фрагменты;
\item модуль получения частотно-временного представления сигнала;
\item модуль извлечения информационных образов;
\item модуль обработки информационных образов;
\item база данных информационных образов;
\item модуль жанровой классификации музыкального произведения.
\item модуль визуализация.
\end{itemize}
Структурная схема, иллюстрирующая перечисленные блоки и связи между ними приведена на чертеже ГУИР.400201.004 C1.
Для решения задачи выделения спектральных, временных и иных признаков из музыкального трека будут использоваться с
ледующие блоки:

\begin{itemize}
\item модуль препроцессинга и нарезки музыкального трека на фрагменты;
\item модуль получения частотно-временного представления сигнала;
\item модуль извлечения информационных образов;
\item модуль обработки информационных образов.
\end{itemize}

Для решения задачи проверки значимости признаков путём использовании их в задаче жанровой классификации будет использоваться модуль жанровой классификации музыкального произведения.

Для решения задачи визуализации данных алгоритмом t-SNE используется модуль визуализации.

Модуль чтения музыкального произведения состоит из двух частей. Первая часть это консольная утилита Lame. Lame свободное приложение для кодирования аудио в формат MP3 (MPEG-1 audio layer 3) и декодирования аудио в WAV формат, который наиболее удобен для чтения и представления музыкальных треков в виде массива. Библиотека Scipy позволяет преобразовать WAV аудио в Numpy массив. 

Модуль препроцессинга обеспечивает первоначальную обработку музыкальных треков. На вход модуля музыкальный трек подаётся как набор сэмплов (NumPy массив). В данном блоке происходит нормализация, фильтрация и, если необходимо, приведения стереозвука к монозвуку .

Модуль получения частотно-временного представления сигнала представляет собой набор преобразований такие как оконное преобразование Фурье и  вейвлеты, которые представляются в виде многомерного NumPy - массива.Так подобные вычисления требуют высокой производительности, то для более эффективного вычисления используются параллельные вычисления на центральном процессоре.

Модуль извлечения информационных образов. Данный модуль представляет из себя набор методов получения информационных образов из временной и частотной области музыкального произведения. Также в этом модуле считаются мел-кепстральные коэффициенты. На вход данному модулю подаётся многомерные Numpy массивы. На выходе получается одномерный Numpy массив информационных признаков для каждого музыкального трека.

Модуль обработки информационных образов. В данном модуле признаки нормализуются по МО и СКО. Удаляются выбросы и некорректные значения.


аза данных информационных образов хранит в себе результат работы всего модуля и хранит в себе набор информационных образов музыкальных треков и принадлежность к тому или иному кластеру.

При выборе базы данных были сформулированы следующие требования:
\begin{itemize}
\item производительность;
\item объектный язык запросов;
\item возможность параллельной записи и чтения.
\end{itemize}
Так как главный модуль является частью сервиса рекомендации музыки, то появляется дополнительные требования к базе данных:
\begin{itemize}
\item маштабируемость;
\item репликация;
\item балансировка нагрузки.
\end{itemize}
С учётом всех этих требований и того факта, что данные для хранения представляют собой простую структуру, выбор пал на базу данных MongoDB.
MongoDB — документоориентированная система управления базами данных (СУБД) с открытым исходным кодом, не требующая описания схемы таблиц. Классифицирована как NoSQL, использует JSON-подобные документы и схему базы данных. Написана на языке C++. Имеется подробная и качественная документация, большое число примеров и драйверов под популярные языки Java, JavaScript, Node.js, C++, 	C\#, PHP, Python, Perl, Ruby.
MongoDB может работать с набором реплик. Набор реплик состоит из двух и более копий данных. Каждый экземпляр набора реплик может в любой момент выступать в роли основной или вспомогательной реплики. Все операции записи и чтения по умолчанию осуществляются с основной репликой. Вспомогательные реплики поддерживают в актуальном состоянии копии данных. В случае, когда основная реплика дает сбой, набор реплик проводит выбор, который из реплик должен стать основным. Второстепенные реплики могут дополнительно является источником для операций чтения. MongoDB масштабируется горизонтально используя шардинг. Пользователь выбирает ключ шарда, который определяет как данные в коллекции будут распределены. Данные разделятюся на диапазоны (в зависимости от ключа шарда) и распределятся по шардам.
Из преимуществ MongoDB:
\begin{enumerate}[label=\arabic*.]
\item Объектный язык запросов.
\item Поддержка индексации.
\item Поддержка Map/Reduce для распределенных операций над данным.
\item Документы, не требующие определения схемы. Одно из самых важных преимуществ. Преимущество заключается в том, что нет нужды хранить пустые ячейки данных в каждом документе.
\item Поддержка сложных массивов. Каждый элемент массива может представлять из себя объект.
\item Поддержка шардинга на уровне платформы.
\item Атомарность гарантируется только на уровне целого документа, то есть частичного обновления документа произойти не может.
\item Любые данные, которые считываются одним клиентом, могут параллельно изменяться другим клиентом.
\end{enumerate}
СУБД управляет наборами JSON-подобных документов, хранимых в двоичном виде в формате BSON. Хранение и поиск файлов в MongoDB происходит благодаря вызовам протокола GridFS.

Модуль жанровой классификации необходим для проверки значимости выделенных образов. Для этого используется набор стандартных алгоритмов классификации из библиотеки Scikit-learn с параметрами по умолчанию. На вход модуля поступает двумерный Numpy-массив. На выходе матрица ошибок. В качестве исходной выборки используется репозиторий музыки GZTAN.

Модуль визуализации визуализирует данные с помощью алгоритма t-SNE. На вход подаётся двумерны Numpy-массив. На выходе изображение в формате JPEG.

Для реализации модулей был выбран язык программирования Python, так как для него существует множество библиотек выполняющих математические расчеты, и облегчающих решение задач связанных с анализом данных и машинным обучением. В работе используются библиотека NumPy - это расширение языка Python, добавляющее поддержку больших многомерных массивов и матриц, вместе с большой библиотекой высокоуровневых математических функций для операций с этими массивами. Также существует библиотека SciKit-learn, которая содержит реализацию алгоритмов машинного обучения инструменты для работы с данными. В дипломной работе также используется библиотека Skipy, которая предназначенная для выполнения научных и инженерных расчётов.

