\begin{titlepage}
\sectioncentered*{РЕФЕРАТ}
\label{sec:ref}

Дипломный проект предоставлен следующим образом. Электронные носители: 1 компакт-диск. Чертежный материал: 6 листов формата А1. Пояснительная записка: 81 страниц, 27 рисунков, 6 таблиц, 18 литературных источников, 4 приложения.

Ключевые слова: информационные признаки, жанровая кассификация, API, визуализация данных.

Объектом исследования и разработки является возможность выделять информационные образы из музыкального произведения для сервиса рекомендации музыки и жанровой классификации.

Целью данного дипломного проекта является создание программного модуля, который выделял информационные образы из музыкального трека только на основании  акустического анализа.

При разработке программного средства использовалась среда разработки PyCharm, pip, консольная утилита Lame, библиотека scipy, библиотека scikit-learn, библиотека matplotlib, язык программирования Python.

Областью практического применения программного средства является сервис рекомендации музыки. Разрабатываемый программный продукт подразумевает собой набор фундаментальных классов и модулей, реализущих взаимодействие с файловой системой и базой данных. 
Данные компоненты интегрируются с логикой получения информационных образов. 

Разработанный программный продукт можно считать экономически эффективным, и он полностью оправдывает вложенные в него средства.

Дипломный проект является завершенным, поставленная задача решена в полной мере, присутствует возможность эффективного расширения программного продукта путем реализации новых компонентов.
\end{titlepage}