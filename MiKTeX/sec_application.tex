\appendix

\appendixtitle{обязательное}{Листинг кода модуля \texttt{cstlstm}}

\begin{center}Класс \texttt{ChildSumTreeLSTMCell}\end{center}

\begin{lstlisting}[style=app]
import json

import numpy as np
import tensorflow as tf
import tensorflow.contrib.eager as tfe


layers = tf.keras.layers


class ChildSumTreeLSTMCell(tf.keras.Model):

    def __init__(self, unit_number, reg_factor, forget_bias=1.0):
        super(ChildSumTreeLSTMCell, self).__init__()
        self.unit_number = unit_number
        self.forget_bias = forget_bias
        self.forget_signals = layers.Dense(unit_number, input_shape=(300 + unit_number,), kernel_regularizer=tf.keras.regularizers.l2(reg_factor))
        self.iou_signals = layers.Dense(unit_number * 3, kernel_regularizer=tf.keras.regularizers.l2(reg_factor))
        self.input_transform = layers.Dense(unit_number, kernel_regularizer=tf.keras.regularizers.l2(reg_factor))

    def call(self, inputs, *args, **kwargs):
        input_word, child_c, child_h = inputs
        child_h_sum = tf.reduce_sum(child_h, 0)
        concat = tf.concat([input_word, child_h_sum], 1)
        i, o, u = tf.split(self.iou_signals(concat), 3, axis=1)
        i, o, u = tf.sigmoid(i), tf.sigmoid(o), tf.tanh(u)
        f = self.forget_signals(child_h) + self.input_transform(input_word) + self.forget_bias
        f = tf.sigmoid(f)
        c = (i * u) + tf.reduce_sum(f * child_c, 0)
        h = o * tf.tanh(c)
        return c, h
\end{lstlisting}

\begin{center}Класс \texttt{ChildSumTreeLSTMClassifier}\end{center}

\begin{lstlisting}[style=app]
class ChildSumTreeLSTMClassifier(tf.keras.Model):

    def __init__(self, embed, unit_numb=150, class_numb=5, reg_factor=0.0001, keep_prob=0.5):
        super(ChildSumTreeLSTMClassifier, self).__init__()
        self.embed = tf.constant(embed)
        self.projection = layers.Dense(self.embed.shape[1], kernel_regularizer=tf.keras.regularizers.l2(reg_factor))
        self.encoder = ChildSumTreeLSTMCell(unit_numb, reg_factor)
        self.output_layer = layers.Dense(class_numb, kernel_regularizer=tf.keras.regularizers.l2(reg_factor))
        self.dropout = layers.Dropout(keep_prob)
        self.reg_factor = reg_factor
        self.loss = 0
        self.result = []
        self.fine = tfe.metrics.Accuracy()
        self.binary = tfe.metrics.Accuracy()
        self.root_fine = tfe.metrics.Accuracy()
        self.root_binary = tfe.metrics.Accuracy()

    def reset_metrics(self):
        self.loss = 0
        self.result = []
        self.fine = tfe.metrics.Accuracy()
        self.binary = tfe.metrics.Accuracy()
        self.root_fine = tfe.metrics.Accuracy()
        self.root_binary = tfe.metrics.Accuracy()

    @property
    def model_variables(self):
        return self.encoder.variables + self.output_layer.variables

    @property
    def embed_variables(self):
        return self.projection.variables

    def _add_metrics(self, label, logits):
        if label == 9:
            return
        self.loss += tf.nn.sparse_softmax_cross_entropy_with_logits(logits=logits[0], labels=label)
        prediction = tf.cast(tf.argmax(logits[0]), tf.int32)
        self.result.append(prediction)
        self.fine(label, prediction)
        if label != 2:
            self.binary(int(label > 2), tf.cast(prediction > 2, tf.int32))

    def call(self, inputs, training=None, *args, **kwargs):
        out = tf.map_fn(lambda tree: self.eval_tree(json.loads(tree.numpy())),
                        inputs,
                        dtype='int32')
        # for tree in inputs:
        #     out = self.eval_tree(json.loads(inputs.numpy()))
        return out

    def eval_tree(self, tree, is_root=True):
        sentiment = tree[0]
        word = tree[1][0]
        childs = tree[1][1]
        if word == -1:
            word_embed = tf.random_uniform((self.embed.shape[1],), -0.05, 0.05)
        else:
            word_embed = tf.nn.embedding_lookup(self.embed, word)
        word_embed = tf.reshape(word_embed, (1, word_embed.shape[0]))
        word_embed = self.projection(word_embed)
        word_embed = self.dropout(word_embed)
        child_h = []
        child_c = []
        for child in childs:
            child_state = self.eval_tree(child, is_root=False)
            child_c.append(child_state[0])
            child_h.append(child_state[1])
        if childs:
            child_c = tf.convert_to_tensor(child_c)
            child_h = tf.convert_to_tensor(child_h)
        else:
            child_c = tf.zeros((1, 1, self.encoder.unit_number))
            child_h = tf.zeros((1, 1, self.encoder.unit_number))
        child_state = self.encoder([word_embed, child_c, child_h])
        logits = self.output_layer(child_state[1])
        self._add_metrics(sentiment, logits)
        if is_root:
            prediction = tf.cast(tf.argmax(logits[0]), tf.int32)
            self.result.append(prediction)
            self.root_fine(sentiment, prediction)
            if sentiment != 2:
                self.root_binary(int(sentiment > 2), tf.cast(prediction > 2, tf.int32))
            return prediction
        return child_state

    def dev_eval(model, epoch, dev_trees, train_loss):
        print('epoch:\%4d, train_loss: \%.3e, dev_loss_avg: \%.3e, dev_accuracy:\n  [\%s]'
              \% (epoch, train_loss, model.loss, evaluate_dataset(model, dev_trees)))

    def checkpoint(self):
        return tfe.Checkpoint(lstm_froget=self.encoder.forget_signals,
                              lstm_iou=self.encoder.iou_signals,
                              lstm_input_transform=self.encoder.input_transform,
                              emdeb_projection=self.projection,
                              output_layer=self.output_layer,
                              dropout=self.dropout)

\end{lstlisting}

\appendixtitle{обязательное}{Листинг кода модуля \texttt{inference}}

\begin{center}Класс \texttt{Inference}\end{center}

\begin{lstlisting}[style=app]
#!/usr/bin/env python3
import sys
import json
import codecs
import numpy as np

import tensorflow as tf
import tensorflow.contrib.eager as tfe
import nltk

from nltk.parse.stanford import StanfordDependencyParser

from labeled_trees import LabeledTree
from cstlstm import ChildSumTreeLSTMClassifier


tf.enable_eager_execution()


def convert_result(tree, results):
    root = LabeledTree()
    if type(tree) == str:
        root.label = results[0][0]
        results[0] = results[0][1:]
        root.text = tree
        return root
    root.text = tree.label()
    for child in tree:
        root.add_child(convert_result(child, results))
    root.label = results[0][0]
    results[0] = results[0][1:]
    return root


def load_embeddings(embedding_path):
    """Loads embedings, returns weight matrix and dict from words to indices."""
    print('loading word embeddings from \%s' \% embedding_path)
    weight_vectors = []
    word2index = {}
    with codecs.open(embedding_path, encoding='utf-8') as f:
        for line in f:
            word, vec = line.split(u' ', 1)
            word2index[word] = len(weight_vectors)
            weight_vectors.append(np.array(vec.split(), dtype=np.float32))
        # Annoying implementation detail; '(' and ')' are replaced by '-LRB-' and
        # '-RRB-' respectively in the parse-trees.
        word2index[u'-LRB-'] = word2index.pop(u'(')
        word2index[u'-RRB-'] = word2index.pop(u')')
        # Random embedding vector for unknown words.
        weight_vectors.append(np.random.uniform(
            -0.05, 0.05, weight_vectors[0].shape).astype(np.float32))
        return np.array(weight_vectors, dtype=np.float32), word2index


def tree_to_list(tree, embed_indexes):
    if type(tree) == str:
        return [0, [embed_indexes[tree], []]]
    root = [0, [embed_indexes[tree.label()], []]]
    for child in tree:
        root[1][1].append(tree_to_list(child, embed_indexes))
    return root


def main():
    checkpoint_directory = sys.argv[1]
    path_to_jar = './data/stanford-parser.jar'
    path_to_models_jar = './data/stanford-parser-3.9.1-models.jar'
    dependency_parser = StanfordDependencyParser(path_to_jar=path_to_jar, path_to_models_jar=path_to_models_jar)
    sentence = "This movie doesn't care about cleverness, wit or any other kind of intelligent humor."#input("Input sentence for sentiment analysis, please:\n")
    tokens = nltk.word_tokenize(sentence)
    tree = dependency_parser.parse_one(tokens)
    tree = tree.tree()

    embed, embed_indexes = load_embeddings('./data/filtered_glove.txt')
    model = ChildSumTreeLSTMClassifier(embed)
    checkpoint = model.checkpoint()
    checkpoint.restore(tf.train.latest_checkpoint(checkpoint_directory))
    json_format = json.dumps(tree_to_list(tree, embed_indexes))

    model(tf.constant([json_format]))
    result = [label.numpy() for label in model.result]
    pretty_tree = convert_result(tree, [result])
    js = "createTrees([{}])\nupdateTrees()".format(pretty_tree.to_json())
    with open('test.js', 'w') as f:
        f.write(js)


if __name__ == "__main__":
    sys.exit(main())
\end{lstlisting}

\appendixtitle{обязательное}{Листинг кода модуля \texttt{labeled\_tree}}

\begin{center}Класс \texttt{LabeledTree}\end{center}

\begin{lstlisting}[style=app]
"""
Make trees visualizable in an IPython notebook
"""
import json

try:
    from PIL import ImageFont
    font = ImageFont.core.getfont("/Library/Fonts/Georgia.ttf", 15)
    def text_size(text):
        return max(4, font.getsize(text)[0][0])
except Exception:
    def text_size(text):
        # TODO(contributors): make changes here to incorporate cap and uncap unknown words.
        return max(4, int(len(text) * 1.1))


class LabeledTree(object):
    SCORE_MAPPING = [-12.5,-6.25,0.0,6.25,12.5]

    def __init__(self,
                 depth=0,
                 text=None,
                 label=None,
                 children=None,
                 parent=None,
                 udepth=1):
        self.label    = label
        self.children = children if children != None else []
        self.general_children = []
        self.text = text
        self.parent   = parent
        self.depth    = depth
        self.udepth   = udepth

    def uproot(tree):
        """
        Take a subranch of a tree and deep-copy the children
        of this subbranch into a new LabeledTree
        """
        uprooted = tree.copy()
        uprooted.parent = None
        for child in tree.all_children():
            uprooted.add_general_child(child)
        return uprooted

    def shrink_tree(tree, final_depth):
        if tree.udepth <= final_depth:
            return tree
        for branch in tree.general_children:
            if branch.udepth == final_depth:
                return branch.uproot()

    def shrunk_trees(tree, final_depth):
        if tree.udepth <= final_depth:
            yield tree
        for branch in tree.general_children:
            if branch.udepth == final_depth:
                yield branch.uproot()

    def copy(self):
        """
        Deep Copy of a LabeledTree
        """
        return LabeledTree(
            udepth = self.udepth,
            depth = self.depth,
            text = self.text,
            label = self.label,
            children = self.children.copy() if self.children != None else [],
            parent = self.parent)

    def add_child(self, child):
        """
        Adds a branch to the current tree.
        """
        self.children.append(child)
        child.parent = self
        self.udepth = max([child.udepth for child in self.children]) + 1

    def add_general_child(self, child):
        self.general_children.append(child)

    def all_children(self):
        if len(self.children) > 0:
            for child in self.children:
                for subchild in child.all_children():
                    yield subchild
            yield self
        else:
            yield self

    def lowercase(self):
        """
        Lowercase all strings in this tree.
        Works recursively and in-place.
        """
        if len(self.children) > 0:
            for child in self.children:
                child.lowercase()
        else:
            self.text = self.text.lower()

    def to_dict(self, index=0):
        """
        Dict format for use in Javascript / Jason Chuang's display technology.
        """
        index += 1
        rep = {}
        rep["index"] = index
        rep["leaf"] = len(self.children) == 0
        rep["depth"] = self.udepth
        rep["scoreDistr"] = [0.0] * len(LabeledTree.SCORE_MAPPING)
        # dirac distribution at correct label
        if self.label is not None:
            rep["scoreDistr"][self.label] = 1.0
            mapping = LabeledTree.SCORE_MAPPING[:]
            rep["rating"] = mapping[self.label] - min(mapping)
        # if you are using this method for printing predictions
        # from a model, the the dot product with the model's output
        # distribution should be taken with this list:
        rep["numChildren"] = len(self.children)
        text = self.text if self.text != None else ""
        seen_tokens = 0
        witnessed_pixels = 0
        for i, child in enumerate(self.children):
            if i > 0:
                text += " "
            child_key = "child\%d" \% (i)
            (rep[child_key], index) = child.to_dict(index)
            text += rep[child_key]["text"]
            seen_tokens += rep[child_key]["tokens"]
            witnessed_pixels += rep[child_key]["pixels"]

        rep["text"] = text
        rep["tokens"] = 1 if (self.text != None and len(self.text) > 0) else seen_tokens
        rep["pixels"] = witnessed_pixels + 3 if len(self.children) > 0 else text_size(self.text)
        return (rep, index)

    def to_json(self):
        rep, _ = self.to_dict()
        return json.dumps(rep)

    def display(self):
        from IPython.display import Javascript, display

        display(Javascript("createTrees(["+self.to_json()+"])"))
        display(Javascript("updateTrees()"))

    def to_lines(self):
        if len(self.children) > 0:
            left_lines, right_lines = self.children[0].to_lines(), self.children[1].to_lines()
            self_line = [left_lines[0] + " " + right_lines[0]]
            return self_line + left_lines + right_lines
        else:
            return [self.text]

    def to_labeled_lines(self):
        if len(self.children) > 0:
            left_lines, right_lines = self.children[0].to_labeled_lines(), self.children[1].to_labeled_lines()
            self_line = [(self.label, left_lines[0][1] + " " + right_lines[0][1])]
            return self_line + left_lines + right_lines
        else:
            return [(self.label, self.text)]

    def __str__(self):
        """
        String representation of a tree as visible in original corpus.

        print(tree)
        #=> '(2 (2 not) (3 good))'

        Outputs
        -------

            str: the String representation of the tree.

        """
        if len(self.children) > 0:
            rep = "(\%d " \% self.label
            for child in self.children:
                rep += str(child)
            return rep + ")"
        else:
            text = self.text\
                .replace("(", "-LRB-")\
                .replace(")", "-RRB-")\
                .replace("{", "-LCB-")\
                .replace("}", "-RCB-")\
                .replace("[", "-LSB-")\
                .replace("]", "-RSB-")

            return ("(\%d \%s) " \% (self.label, text))

    @staticmethod
    def inject_visualization_javascript(tree_width=1200, tree_height=400, tree_node_radius=10):
        """
        In an Ipython notebook, show SST trees using the same Javascript
        code as used by Jason Chuang's visualisations.
        """
        from .javascript import insert_sentiment_markup
        insert_sentiment_markup(tree_width=tree_width, tree_height=tree_height, tree_node_radius=tree_node_radius)
\end{lstlisting}

\appendixtitle{обязательное}{Листинг кода модуля \texttt{train}}

\begin{center}Класс \texttt{Treebank}\end{center}

\begin{lstlisting}[style=app]
#!/usr/bin/env python3

import json
import os
import sys
import time
import codecs

import numpy as np
import tensorflow as tf
import tensorflow.contrib.eager as tfe

from cstlstm import ChildSumTreeLSTMClassifier

tf.enable_eager_execution()


class Treebank(tf.data.TextLineDataset):

    def __init__(self, data_dir, set_name, embed_indexes):
        json_path = os.path.join(data_dir, 'json', set_name + '.json')
        set_path = os.path.join(data_dir, set_name)
        with codecs.open(json_path, encoding='utf-8') as f:
            trees = json.load(f)
        trees = [replace_words_by_index(tree, embed_indexes) for tree in trees]
        if set_name == 'train':
            trees = [tree for tree in trees if tree[0] != 2]
        self.size = len(trees)
        jsons = ''.join([json.dumps(tree) + '\n' for tree in trees])
        with open(set_path, 'w') as f:
            f.write(jsons)
        super(Treebank, self).__init__(set_path)

    def print_progress_bar(self, iteration):
        """
        Call in a loop to create terminal progress bar
        @params:
            iteration   - Required  : current iteration (Int)
            fill        - Optional  : bar fill character (Str)
        """
        decimals = 1
        fill = '█'
        length = 80
        percent = ("{0:." + str(decimals) + "f}").format(100 * (iteration / float(self.size)))
        filledLength = int(length * iteration // self.size)
        bar = fill * filledLength + '-' * (length - filledLength)
        print('\rProgress: |\%s| \%s\%\% Complete' \% (bar, percent), end='\r')
        # Print New Line on Complete
        if iteration >= self.size:
            print()

\end{lstlisting}

\begin{center}Класс \texttt{Train}\end{center}

\begin{lstlisting}[style=app]

def load_embeddings(embedding_path):
    """Loads embedings, returns weight matrix and dict from words to indices."""
    print('loading word embeddings from \%s' \% embedding_path)
    weight_vectors = []
    word2index = {}
    with codecs.open(embedding_path, encoding='utf-8') as f:
        for line in f:
            word, vec = line.split(u' ', 1)
            word2index[word] = len(weight_vectors)
            weight_vectors.append(np.array(vec.split(), dtype=np.float32))
        # Annoying implementation detail; '(' and ')' are replaced by '-LRB-' and
        # '-RRB-' respectively in the parse-trees.
        word2index[u'-LRB-'] = word2index.pop(u'(')
        word2index[u'-RRB-'] = word2index.pop(u')')
        # Random embedding vector for unknown words.
        weight_vectors.append(np.random.uniform(
            -0.05, 0.05, weight_vectors[0].shape).astype(np.float32))
        return np.array(weight_vectors, dtype=np.float32), word2index


def evaluate_dataset(model, trees, batch_size=25):
    model.reset_metrics()
    complicity = 0
    for batch in tfe.Iterator(trees.shuffle(buffer_size=12000).batch(batch_size)):
        model(batch)
        complicity += batch_size
        trees.print_progress_bar(complicity)
    print('\n')
    metrics = {name: metric.result().numpy() * 100 for name, metric in zip(["fine", "binary", "root_fine", "root_binary"],
                                                                           [model.fine, model.binary, model.root_fine, model.root_binary])}
    return ' '.join(['\%s: \%.2f' \% (name, metric) for name, metric in metrics.items()])


def dev_eval(model, epoch, dev_trees, train_loss):
    print('epoch:\%4d, train_loss: \%.3e, dev_loss_avg: \%.3e, dev_accuracy:\n  [\%s]'
          \% (epoch, train_loss, model.loss, evaluate_dataset(model, dev_trees)))


def train_epoch(model, train_trees, model_optimizer, embedding_optimizer, batch_size=25):
    train_loss = 0
    model.reset_metrics()
    complicity = 0
    for batch in tfe.Iterator(train_trees.shuffle(buffer_size=12000).batch(batch_size)):
        with tfe.GradientTape() as tape:
            tape.watch(model.variables)
            model(batch)
        gradient = tape.gradient(model.loss, model.variables)
        embed_grads = gradient[:len(model.embed_variables)]
        model_grads = gradient[len(model.embed_variables):]
        train_loss += model.loss
        model_optimizer.apply_gradients(zip(model_grads, model.model_variables), global_step=tf.train.get_global_step())
        embedding_optimizer.apply_gradients(zip(embed_grads, model.embed_variables), global_step=tf.train.get_global_step())
        complicity += batch_size
        train_trees.print_progress_bar(complicity)
    print('\n')
    return train_loss


def replace_words_by_index(tree, word2index):
    tree[0] = int(tree[0])
    tree[1][0] = word2index.get(tree[1][0], -1)
    tree[1][1] = [replace_words_by_index(child, word2index) for child in tree[1][1]]
    return tree


def main():
    epochs = 5
    learning_rate = 0.05
    embedding_learning_rate_factor = 0.1
    checkpoint_directory = "./data/checkpoints"
    checkpoint_prefix = os.path.join(checkpoint_directory, "cstlstm")

    embed, embed_indexes = load_embeddings('./data/filtered_glove.txt')
    train_trees = Treebank('./data', 'train', embed_indexes)
    dev_trees = Treebank('./data', 'dev', embed_indexes)
    test_trees = Treebank('./data', 'test', embed_indexes)

    model = ChildSumTreeLSTMClassifier(embed)
    model_optimizer = tf.train.AdagradOptimizer(learning_rate)
    embedding_optimizer = tf.train.AdagradOptimizer(learning_rate * embedding_learning_rate_factor)
    checkpoint = model.checkpoint()

    if len(sys.argv) > 1 and sys.argv[1] == 'skip_train':
        checkpoint.restore(tf.train.latest_checkpoint(checkpoint_directory))
        print("Test results: {}".format(evaluate_dataset(model, test_trees)))
        return 0

    try:
        for epoch in range(epochs):
            start = time.time()
            train_loss = train_epoch(model, train_trees, model_optimizer, embedding_optimizer)
            print("time spent {}".format(time.time() - start))
            checkpoint = model.checkpoint()
            checkpoint.save(file_prefix=checkpoint_prefix)
            dev_eval(model, epoch, dev_trees, train_loss)
    except KeyboardInterrupt:
        print("\nInterrupt training procces")
    print("Test results: {}".format(evaluate_dataset(model, test_trees)))
    return 0


if __name__ == "__main__":
    sys.exit(main())
\end{lstlisting}

\appendixtitle{обязательное}{Листинг кода модуля \texttt{model}}

\begin{center}Класс \texttt{models.SubCategory}\end{center}

\begin{lstlisting}[style=app]

from sqlalchemy.dialects.mysql import VARCHAR
from src.models import sa, orm, Base, Category, CategoryRelation, DeclarativeBase

class SubCategory(Base, DeclarativeBase):
    """The model for SubCategory data"""
    __tablename__ = "sub_category"

    description = sa.Column(VARCHAR(100, charset='utf8'))
    category = orm.relationship("Category", back_popuflates="sub_categories")

\end{lstlisting}

\begin{center}Класс \texttt{models.Category}\end{center}

\begin{lstlisting}[style=app]

from sqlalchemy.dialects.mysql import VARCHAR
from src.models import sa, orm, Base, CategoryRelation, DeclarativeBase

class Category(Base, DeclarativeBase):
    """The model for Category data"""
    __tablename__ = "category"

    description = sa.Column(VARCHAR(100, charset='utf8'))

\end{lstlisting}

\begin{center}Класс \texttt{models.SubCategoryLink}\end{center}

\begin{lstlisting}[style=app]

from sqlalchemy.schema import ForeignKey
from src.models import sa, orm, DeclarativeBase

class CategoryRelation(DeclarativeBase):
    """The model for SubCategoryLink data"""
    __tablename__ = "category_relation"

    id = sa.Column(sa.Integer, primary_key=True)
    item_id = sa.Column(sa.Integer, ForeignKey("item.id"))
    sub_category_id = sa.Column(sa.Integer, ForeignKey("sub_category.id"))
    sub_category = orm.relationship("SubCategory", back_popuflates="items")
    item = orm.relationship("Kernel", back_populates="sub_categories")

\end{lstlisting}

\begin{center}Класс \texttt{models.Item}\end{center}

\begin{lstlisting}[style=app]

from sqlalchemy.schema import ForeignKey
from src.models import sa, orm, DeclarativeBase

class Item(DeclarativeBase):
    """The model for Item data"""
    __tablename__ = "item"

    id = sa.Column(sa.Integer, primary_key=True)
    name = sa.Column(VARCHAR(100, charset='utf8'))
    cost = sa.Column(sa.Integer, primary_key=True)
    rating = sa.Column(sa.Integer, primary_key=True)
    state = sa.Column(sa.Integer, primary_key=True)
    number = sa.Column(sa.Integer, primary_key=True)
    sold = sa.Column(sa.Integer, primary_key=True)
    seller = orm.relationship("Seller", back_popuflates="items")
    sstore = orm.relationship("Store", back_popuflates="items")

\end{lstlisting}

\begin{center}Класс \texttt{models.Store}\end{center}

\begin{lstlisting}[style=app]

from sqlalchemy.schema import ForeignKey
from src.models import sa, orm, DeclarativeBase

class Store(DeclarativeBase):
    """The model for Store data"""
    __tablename__ = "store"

    id = sa.Column(sa.Integer, primary_key=True)
    name = sa.Column(VARCHAR(100, charset='utf8'))

\end{lstlisting}

\appendixtitle{обязательное}{Листинг кода модуля \texttt{downloader}}

\begin{center}Класс \texttt{Downloader}\end{center}

\begin{lstlisting}[style=app]
#!/usr/bin/env python3

import os
import requests

basedir, _ = os.path.split(__file__)

# logging
import logging, logging.config
log_config_file = os.path.join(basedir, 'logging.conf')
if os.path.exists(log_config_file):
    # load config from config file, see logging.conf for configuration settings
    logging.config.fileConfig(log_config_file)
else:
    # or just do a basic config
    format = '\%(name)s - \%(levelname)s - \%(filename)s - \%(lineno)d - \%(message)s'
    logging.basicConfig(format=format, level=logging.NOTSET)

log = logging.getLogger(__package__)
log.debug("logging is setup")

import src.models as db
log.debug("db import")


class BaseApp():

    def __init__(self):
        log.debug('start init')

        db_driver = 'mysql+oursql'
        db_url = db.sa.engine.url.URL(db_driver,
                                      host='localhost',
                                      database='kaggle',
                                      username='root')
        db_url = str(db_url) + '?charset=utf8'
        log.debug("db: \%s\n\n" \% (db_url))

        log.debug("connect db")
        engine = db.sa.create_engine(db_url, echo=False)
        db.init_model(engine)
        self.session = db.db_session()

        log.debug("setup db")
        db.metadata.create_all(engine)
        log.debug("db created")

        log.debug('end init')

    def get_kernels(self):
        TOKEN_LEN = 10
        params = {
            'sortBy': 'creation',
            'group': 'everyone',
            'pageSize': TOKEN_LEN,
        }
        self.get_missing(params)

    def get_competitions(self):
        competitions_url = 'https://www.kaggle.com/competitions.json'
        PAGE_SIZE = 20
        params = {'sortBy': 'recentlyCreated',
                  'group': 'general',
                  'page': 1,
                  'pageSize': PAGE_SIZE}
        response = requests.get(competitions_url, params)
        total_number = response.json()['pagedCompetitionGroup']['totalCompetitions']
        for page in range(1, (total_number // PAGE_SIZE) + 2):
            params['page'] = page
            response = requests.get(competitions_url, params)
            competitions = response.json()['pagedCompetitionGroup']['competitions']
            for competition in competitions:
                self.save_competition_info(competition)

    def save_competition_info(self, competition_json):
        TOKEN_LEN = 10
        competition = self.session.query(db.Competition).filter(db.Competition.id == competition_json['competitionId']).first()
        if not competition:
            competition = db.Competition(id=competition_json['competitionId'],
                                         title=competition_json['competitionTitle'])
            self.session.add(competition)
            self.session.flush()
            self.session.commit()

        cats = self.get_categories(competition_json)
        for cat in cats:
            rel = db.CategoryRelation(category=cat,
                                      competition=competition)
            self.session.add(rel)
        params = {
            'sortBy': 'creation',
            'group': 'everyone',
            'pageSize': TOKEN_LEN,
            'competitionId': competition.id
        }
        self.get_missing(params)

    def get_missing(self, params):
        kernel_list_url = 'https://www.kaggle.com/kernels.json'
        while True:
            try:
                response = requests.get(kernel_list_url, params)
            except Exception as error:
                log_file = open('log.txt', "a")
                log_file.write("Crushed on GET {} with {} with {}".format(kernel_list_url,
                                                                          params,
                                                                          error))
                log_file.write('Exception: {}\n'.format(type(error)))
                log_file.close()
                continue
            if not response.json():
                return
            params['after'] = response.json()[-1]['id']
            for kernel in response.json():
                db_kernel = self.session.query(db.Kernel).filter(db.Kernel.id == kernel['id']).first()
                competition_id = None
                if db_kernel and db_kernel.source_version != kernel['scriptVersionId']:
                    competition_id = db_kernel.competition_id
                    self.session.delete(db_kernel)
                    db_kernel = None
                if not db_kernel:
                    newbie = self.save_kernel_info(kernel)
                    if competition_id is not None:
                        newbie.competition_id = competition_id
                    elif 'competitionId' in params:
                        newbie.competition_id = params['competitionId']
            self.session.flush()
            self.session.commit()

    def save_kernel_info(self, kernel_json):
        log_msg = 'Try to process kernel id: {} lang: {} notebook: {} scr_id: {}'
        log_msg = log_msg.format(kernel_json['id'],
                                 kernel_json['aceLanguageName'],
                                 kernel_json['isNotebook'],
                                 kernel_json['scriptVersionId'])
        log.debug(log_msg)
        author_id = kernel_json['author']['userId']
        author = self.session.query(db.User).filter(db.User.id == author_id).first()
        if not author:
            author = db.User(id=author_id,
                             title=kernel_json['author']['displayName'],
                             user_name=kernel_json['author']['userName'],
                             tier=kernel_json['author']['tier'])
        new_kernel = db.Kernel(id=kernel_json['id'],
                               title=kernel_json['title'],
                               lang=kernel_json['aceLanguageName'],
                               notebook=kernel_json['isNotebook'],
                               votes=kernel_json['totalVotes'],
                               best_score=kernel_json['bestPublicScore'],
                               source_version=kernel_json['scriptVersionId'],
                               author=author)
        download_url = 'https://www.kaggle.com/kernels/sourceurl/{}'
        download_url = download_url.format(new_kernel.source_version)
        try:
            response = requests.get(download_url)
            download_url = response.content.decode('utf-8')
            techs = self.get_technology(download_url,
                                        new_kernel.lang,
                                        new_kernel.notebook)
            for tech in techs:
                rel = db.TechnologyRelation(technology=tech,
                                            kernel=new_kernel)
                self.session.add(rel)
        except Exception as error:
            log_file = open('log.txt', "a")
            log_file.write(log_msg + ' ')
            log_file.write("Crushed on GET {} with {}\n".format(download_url, error))
            log_file.write('Exception: {}\n'.format(type(error)))
            log_file.close()
        cats = self.get_categories(kernel_json)
        for cat in cats:
            rel = db.CategoryRelation(category=cat,
                                      kernel=new_kernel)
            self.session.add(rel)
        data_links = self.get_data_link(kernel_json)
        for link in data_links:
            rel = db.DataLinkRelation(data_link=link,
                                      kernel=new_kernel)
            self.session.add(rel)
        self.session.add(new_kernel)
        return new_kernel

    def get_categories(self, kaggle_object):
        cat_dict = {cat['name']: cat for cat in kaggle_object['categories']['categories']}
        if not cat_dict:
            return []
        old = self.session.query(db.Category).filter(db.Category.title.in_(cat_dict.keys())).all()
        old_names = {cat.title for cat in old}
        new_names = set(cat_dict.keys()) - old_names
        new_categories = []
        for name in new_names:
            cat = cat_dict[name]
            new_categories.append(db.Category(title=cat['name'],
                                              description=cat['description']))
        self.session.add_all(new_categories)
        return new_categories + old

    def get_technology(self, src_url, lang, notebook):
        script = os.path.join(basedir, 'get_tech.sh')
        script += ' \"{}\" \"{}\" \"{}\"'.format(src_url, lang, notebook)
        names = os.popen(script).read().split()
        if not names:
            return names
        old = self.session.query(db.Technology).filter(db.Technology.title.in_(names)).all()
        old_names = [tech.title for tech in old]
        new_names = set(names) - set(old_names)
        new_tech = [db.Technology(title=title) for title in new_names]
        self.session.add_all(new_tech)
        return old + new_tech

    def get_data_link(self, kernel):
        src_dict = {src['name']: src for src in kernel['dataSources']}
        if not src_dict:
            return []
        old = self.session.query(db.DataLink).filter(db.DataLink.title.in_(src_dict.keys())).all()
        old_names = {src.title for src in old}
        new_names = set(src_dict.keys()) - old_names
        new_data_links = []
        for name in new_names:
            src = src_dict[name]
            new_data_links.append(db.DataLink(title=src['name'],
                                              link=src['url']))
        self.session.add_all(new_data_links)
        return new_data_links + old

    def run(self):
        # self.get_kernels()
        self.get_competitions()

\end{lstlisting}

\begin{center}Класс \texttt{models.Achievement}\end{center}

\begin{lstlisting}[style=app]

from sqlalchemy.schema import ForeignKey
from src.models import sa, orm, DeclarativeBase

class Achievement(DeclarativeBase):
    """The model for Achievement data"""
    __tablename__ = "achievement"

    id = sa.Column(sa.Integer, primary_key=True)
    name = sa.Column(VARCHAR(100, charset='utf8'))

\end{lstlisting}

\begin{center}Класс \texttt{models.AchievementLink}\end{center}

\begin{lstlisting}[style=app]

from sqlalchemy.schema import ForeignKey
from src.models import sa, orm, DeclarativeBase

class AchievementLink(DeclarativeBase):
    """The model for AchievementLink data"""
    __tablename__ = "achievement_link"

    id = sa.Column(sa.Integer, primary_key=True)
    achievement = orm.relationship("Achievement", back_popuflates="item_link")
    item = orm.relationship("Kernel", back_populates="achievement_link")

  \end{lstlisting}


\begin{center}Класс \texttt{models.Genre}\end{center}

\begin{lstlisting}[style=app]

from sqlalchemy.schema import ForeignKey
from src.models import sa, orm, DeclarativeBase

class Genre(DeclarativeBase):
    """The model for Genre data"""
    __tablename__ = "genre"

    id = sa.Column(sa.Integer, primary_key=True)
    name = sa.Column(VARCHAR(100, charset='utf8'))

\end{lstlisting}

\begin{center}Класс \texttt{models.GenreLink}\end{center}

\begin{lstlisting}[style=app]

from sqlalchemy.schema import ForeignKey
from src.models import sa, orm, DeclarativeBase

class GenreLink(DeclarativeBase):
    """The model for GenreLink data"""
    __tablename__ = "genre_link"

    id = sa.Column(sa.Integer, primary_key=True)
    genre = orm.relationship("Genre", back_popuflates="film_link")
    film = orm.relationship("Film", back_populates="genre_link")

  \end{lstlisting}
