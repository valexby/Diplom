\sectioncentered*{Заключение}
\addcontentsline{toc}{section}{ЗАКЛЮЧЕНИЕ}
\setcounter{page}{81}
\label{sec:outro}
В результате выполнения дипломного проекта была разработана система анализа тональностей в тексте. Данное приложение предназначено для сервисов предлагающих товары и услуги множеству пользователей.

Программный продукт возможно адаптировать, для того чтобы автономно собирать информацию о продуктах и услугах, и получать отзывы о них от пользователей. После этого модель нейронной сети позволяет классифицировать данные отзывы по настроениям пользователей. Это помогает выделить какие отзывы требуют внимательного изучения работниками сервиса, а какие из них них не играют большой роли для маркетинга.

Особенностью, которая отличает данное приложение от аналогов, является гибкость системы, и широкие возможности в визуализации результатов анализа. Это достигнуто за счет использования усложненной структуры нейронной сети, что значительно сокращает время ее обучения, и предоставляет пользователю информацию о том, почему модель сделала тот или иной вывод для отзыва. Данная особенность системы дает возможность настроить ее для использования в сервисах, где специфические термины и лексика играют важную роль в семантике отзывов.

Вся собранная и обработанная информация хранится в базе данных, что представляет возможности для регрессионого анализа товаров и услуг.

На основании вышеприведенных сведений, поставленные цели можно считать выполненными в полном объеме. Дальнейшие планы расширения приложения заключаются в увеличении производительности приложения за счет оптимизации оптимизации обработки групп наборов данных. А так же планируется добавление графического интерфейса, что позволит без особых усилий управлять системой.