\sectioncentered*{Введение}\label{sec:intro}
\addcontentsline{toc}{section}{Введение}

В настоящее время машинное обучение нашло практическое применение в двух сферах: обработка изображений и обработка естественных языков. Обработка изображений более популярна и в то же время принципиально проще. Способностью к распознаванию образов обладают существа, которых мы бы назвали примитивными. Хотя конечно, если сравнивать их сложность с нейронными сетями, то примитивными окажутся сети. Полноценной речью же обладают лишь люди, и никто из известных нам существ не имеет более развитой системы сигналов. Даже для людей речь оказывается сложнее, чем распознавание образов. Дети учатся говорить намного позже, чем различать образы.

Распознавание речи так же делится на несколько направлений: перенос живой речи в текст и наоборот, классификация текстов, машинный перевод. Первые успешные опыты по распознаванию речи были проведены еще в середине прошлого столетия, однако динамика роста производительности алгоритмов была невелика. Однако в этом десятилетии значимые открытия происходят в области каждый месяц. Это происходит потому что теоретическая база, созданная десятки лет назад, только сейчас может быть эффективно применена с использованием современного оборудования.

На данный момент самые популярные задачи на рынке --- это задачи классификации текстов. Рынки машинного перевода, поисковых систем, обработки живой речи сосредоточены в нескольких громадных проектов мировых корпораций. Эти задачи крайне затратны по ресурсам и небольшая кампания или стартап не могут себе позволить реализовать собственные поисковые системы, или машинный перевод и вынуждены пользоваться внешними сервисами. В то время, как небольшой стартап интернет-магазина может применять какое-то время внешний сервис для анализа тональностей в отзывах покупателей. Для интернет-магазина учитывать пожелания и мнения пользователей --- это жизненно важная задача. Однако получив в какой-то момент достаточный уровень финансирования нередко кампания берется за интеграцию собственной системы анализа естественных языков.

Задачи классификации дешевле за счет принципиальной разницы в проблемах, которые необходимо решить в данных областях чтобы получить качественный результат. Для машинного перевода главной проблемой является качественное кодирование слов, то есть необходимы большие мощности для обработки больших массивов данных. Для поисковых систем необходимы еще большие мощности, для постоянной индексации пространства поиска. При классификации текста основной задачей является понимание семантики текста. Математически эта задача сложнее остальных. Однако в ответ на сложную задачу строятся сложные структуры, которые намного лучше обучаются.

Целью данного дипломного проекта является реализация приложения, применяющее методы машинного обучения для анализа отзывов о фильмах пользователей онлайн-форума.

В ходе данной работы была реализована модель Tree LSTM, основная идея которой в том, чтобы обрабатывать текст согласно его синтаксической структуре. Если представить развертку во времени в виде синтаксического дерева, то модель нейронной сети будет представлять собой многослойную нейронную сеть, чья структура меняется в зависимости от входных данных. Такая сложная динамическая структура позволяется сети обучаться понимать семантику без серьезных требований к оборудованию.

Для демонстрации возможностей системы был разработан модуль, собирающий отзывы к фильмам. Эти отзывы классифицируются и оценки тональности сравниваются с оценками поставленными авторами отзывов. Однако, собранная информация о фильмах предоставляет возможности для различных исследований, так как реализованную модель можно обучить классификации не только по уровням тональности, но и по каким-то другим свойствам текста.

В ходе данного проектирования были выполнены следующие задачи:
\begin{itemize}
\item исследование методов машинного обучения для обработки текстов;
\item разработка структурной схемы системы;
\item выбор современный и эффективных технологий в области;
\item реализация модели нейронной сети;
\item реализация модуля сбора статистики;
\item реализация модулей поддержки и визуализации.
\end{itemize}