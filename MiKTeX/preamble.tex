% !TeX spellcheck = russian-aot-ieyo
% Зачем: Определяет класс документа (То, как будет выглядеть документ)
% Примечание: параметр draft помечает строки, вышедшие за границы страницы, прямоугольником, в фильной версии его нужно удалить.
\documentclass[a4paper,14pt,russian,oneside,final]{extreport}

% Зачем: Настройка Times New Roman.
% Рекомендовано для Windows (нужен PSCyr, подробности см. в fonts_windows.tex)
% раскомментировать, чтобы использовать:
\usepackage{fontspec}
\usepackage{xltxtra}


\usepackage{polyglossia}

\setmainlanguage[babelshorthands=true]{russian}
\setotherlanguage{english}
\setmonofont{Courier New}
\newfontfamily\cyrillicfonttt{Courier New}[Script = Cyrillic]
\defaultfontfeatures{Ligatures=TeX}
\setmainfont{Arial}
\newfontfamily\cyrillicfont{Times New Roman}[Script = Cyrillic]
\setsansfont{Times New Roman}
\newfontfamily\cyrillicfontsf{Arial}[Script = Cyrillic]

\newfontfamily\englishfontsf{Arial}[Script = Latin]
\newfontfamily\englishfont{Times New Roman}[Script = Latin]
\newfontfamily\englishfonttt{Courier New}[Script = Latin]


\newcommand{\en}[1]{\foreignlanguage{english}{#1}}
% не забудьте закомментировать % Зачем: Выбор внутренней TeX кодировки.
\usepackage[T2A]{fontenc}

% Зачем: Предоставляет свободный Times New Roman.
% Шрифт идёт вместе с пакетом scalable-cyrfonts-tex в Ubuntu/Debian

% пакет scalable-cyrfonts-tex может конфликтовать с texlive-fonts-extra в Ubuntu
% решение: Для себя я решил эту проблему так: пересобрал пакет scalable-cyrfonts-tex, изменив его имя. Решение топорное, но работает. Желающие могут скачать мой пакет здесь:
% https://yadi.sk/d/GW2PhDgEcJH7m
% Установка:
% dpkg -i scalable-cyrfonts-tex-shurph_4.16_all.deb

\usefont{T2A}{ftm}{m}{sl}


% Рекомендовано для Linux (нужен scalable-cyrfonts-tex, подробности см. в fonts_linux.tex)
% раскомментировать, чтобы использовать:
%% Зачем: Выбор внутренней TeX кодировки.
\usepackage[T2A]{fontenc}

% Зачем: Предоставляет свободный Times New Roman.
% Шрифт идёт вместе с пакетом scalable-cyrfonts-tex в Ubuntu/Debian

% пакет scalable-cyrfonts-tex может конфликтовать с texlive-fonts-extra в Ubuntu
% решение: Для себя я решил эту проблему так: пересобрал пакет scalable-cyrfonts-tex, изменив его имя. Решение топорное, но работает. Желающие могут скачать мой пакет здесь:
% https://yadi.sk/d/GW2PhDgEcJH7m
% Установка:
% dpkg -i scalable-cyrfonts-tex-shurph_4.16_all.deb

\usefont{T2A}{ftm}{m}{sl}
% не забудьте закомментировать \usepackage{fontspec}
\usepackage{xltxtra}


\usepackage{polyglossia}

\setmainlanguage[babelshorthands=true]{russian}
\setotherlanguage{english}
\setmonofont{Courier New}
\newfontfamily\cyrillicfonttt{Courier New}[Script = Cyrillic]
\defaultfontfeatures{Ligatures=TeX}
\setmainfont{Arial}
\newfontfamily\cyrillicfont{Times New Roman}[Script = Cyrillic]
\setsansfont{Times New Roman}
\newfontfamily\cyrillicfontsf{Arial}[Script = Cyrillic]

\newfontfamily\englishfontsf{Arial}[Script = Latin]
\newfontfamily\englishfont{Times New Roman}[Script = Latin]
\newfontfamily\englishfonttt{Courier New}[Script = Latin]


\newcommand{\en}[1]{\foreignlanguage{english}{#1}}


% Зачем: Установка кодировки исходных файлов.
\usepackage[utf8]{inputenc}

% Зачем: Делает результирующий PDF "searchable and copyable".
\usepackage{cmap}

% Зачем: Чтобы можно было использовать русские буквы в формулах, но в случае использования предупреждать об этом.
\usepackage[warn]{mathtext}

% Зачем: Учет особенностей различных языков.
% \usepackage[russian]{babel}

% Зачем: Добавляет поддержу дополнительных размеров текста 8pt, 9pt, 10pt, 11pt, 12pt, 14pt, 17pt, and 20pt.
% Почему: Пункт 2.1.1 Требований по оформлению пояснительной записки.
\usepackage{extsizes}

% Зачем: Длинна, пимерно соответвующая 5 символам
% Почему: Требования содержат странное требование про отсупы в 5 символов (для немоноширинного шрифта :| )
\newlength{\fivecharsapprox}
\setlength{\fivecharsapprox}{1.25cm}


% Зачем: Добавляет отступы для абзацев.
% Почему: Пункт 2.1.3 Требований по оформлению пояснительной записки.
\usepackage{indentfirst}
\setlength{\parindent}{\fivecharsapprox} % Примерно соответсвует 5 символам.


% Зачем: Настраивает отступы от границ страницы.
% Почему: Пункт 2.1.2 Требований по оформлению пояснительной записки.
\usepackage[left=3cm,top=2.0cm,right=1.5cm,bottom=2.695cm]{geometry}


% Зачем: Настраивает межстрочный интервал, для размещения 40 +/- 3 строки текста на странице.
% Почему: Пункт 2.1.1 Требований по оформлению пояснительной записки.
\usepackage[nodisplayskipstretch]{setspace}
\setstretch{1.0}
%\onehalfspacing

% Зачем: Выбор шрифта по-умолчанию.
% Почему: Пункт 2.1.1 Требований по оформлению пояснительной записки.
% Примечание: В требованиях не указан, какой именно шрифт использовать. По традиции используем TNR.
\renewcommand{\rmdefault}{ftm} % Times New Roman


% Зачем: Отключает использование изменяемых межсловных пробелов.
% Почему: Так не принято делать в текстах на русском языке.
\frenchspacing


% Зачем: Сброс счетчика сносок для каждой страницы
% Примечание: в "Требованиях по оформлению пояснительной записки" не указано, как нужно делать, но в других БГУИРовских докуметах рекомендуется нумерация отдельная для каждой страницы
\usepackage{perpage}
\MakePerPage{footnote}


% Зачем: Добавляет скобку 1) к номеру сноски
% Почему: Пункты 2.9.2 и 2.9.1 Требований по оформлению пояснительной записки.
\makeatletter
\def\@makefnmark{\hbox{\@textsuperscript{\normalfont\@thefnmark)}}}
\makeatother


% Зачем: Расположение сносок внизу страницы
% Почему: Пункт 2.9.2 Требований по оформлению пояснительной записки.
\usepackage[bottom]{footmisc}

% Зачем: Пункты (в терминологии требований) в терминологии TeX subsubsection должны нумероваться
% Почему: Пункт 2.2.3 Требований по оформлению пояснительной записки.
\setcounter{secnumdepth}{3}


% Зачем: Настраивает отступ между таблицей с содержанимем и словом СОДЕРЖАНИЕ
% Почему: Пункт 2.2.7 Требований по оформлению пояснительной записки.
\usepackage{tocloft}
\setlength{\cftbeforetoctitleskip}{-1em}
\setlength{\cftaftertoctitleskip}{1em}

% Зачем: Определяет отступы слева для записей в таблице содержания.
% Почему: Пункт 2.2.7 Требований по оформлению пояснительной записки.
\makeatletter
\renewcommand{\l@section}{\@dottedtocline{1}{0.5em}{1.2em}}
\renewcommand{\l@subsection}{\@dottedtocline{2}{1.7em}{2.0em}}

\makeatother


% Зачем: Работа с колонтитулами
\usepackage{fancyhdr} % пакет для установки колонтитулов
\pagestyle{fancy} % смена стиля оформления страниц


% Зачем: Нумерация страниц располагается справа снизу страницы
% Почему: Пункт 2.2.8 Требований по оформлению пояснительной записки.
\fancyhf{} % очистка текущих значений
\fancyfoot[R]{\thepage} % установка верхнего колонтитула
\renewcommand{\footrulewidth}{0pt} % убрать разделительную линию внизу страницы
\renewcommand{\headrulewidth}{0pt} % убрать разделительную линию вверху страницы
\fancypagestyle{plain}{
    \fancyhf{}
    \rfoot{\thepage}}


% Зачем: Задает стиль заголовков раздела жирным шрифтом, прописными буквами, без точки в конце
% Почему: Пункты 2.1.1, 2.2.5, 2.2.6 и ПРИЛОЖЕНИЕ Л Требований по оформлению пояснительной записки.
\makeatletter
\renewcommand\thesubsection{\thesection.\arabic{subsection}}
\renewcommand\thesection{\arabic{section}}
\makeatother

\makeatletter
\renewcommand\section{%
  \clearpage\@startsection {section}{1}%
    {\fivecharsapprox}%
    {-1em \@plus -1ex \@minus -.2ex}%
    {1em \@plus .2ex}%
    {\raggedright\hyphenpenalty=10000\normalfont\MakeUppercase}}
\makeatother

% Зачем: Задает стиль заголовков подразделов
% Почему: Пункты 2.1.1, 2.2.5 и ПРИЛОЖЕНИЕ Л Требований по оформлению пояснительной записки.
\makeatletter
\renewcommand\subsection{%
  \@startsection{subsection}{2}%
    {\fivecharsapprox}%
    {-1em \@plus -1ex \@minus -.2ex}%
    {1em \@plus .2ex}%
    {\raggedright\hyphenpenalty=10000\normalfont\normalsize}}
\makeatother


% Зачем: Задает стиль заголовков пунктов
% Почему: Пункты 2.1.1, 2.2.5 и ПРИЛОЖЕНИЕ Л Требований по оформлению пояснительной записки.
\makeatletter
\renewcommand\subsubsection{
  \@startsection{subsubsection}{3}%
    {\fivecharsapprox}%
    {-1em \@plus -1ex \@minus -.2ex}%
    {1em \@plus .2ex}%
    {\raggedright\hyphenpenalty=10000\normalfont\normalsize}}
\makeatother

% Зачем: для оформления введения и заключения, они должны быть выровнены по центру.
% Почему: Пункты 1.1.15 и 1.1.11 Требований по оформлению пояснительной записки.
\makeatletter
\newcommand\sectioncentered{%
  \clearpage\@startsection {section}{1}%
    {\z@}%
    {-1em \@plus -1ex \@minus -.2ex}%
    {1em \@plus .2ex}%
    {\centering\hyphenpenalty=10000\normalfont\MakeUppercase}%
    }
\makeatother

\usepackage{titlesec}
\titleformat{\section}
  {}{\textbf{\thesection}}{1em}{}
\titleformat{\subsection}
  {}{\textbf{\thesubsection}}{1em}{}
\titleformat{\subsubsection}
  {}{\textbf{\thesubsubsection}}{1em}{}

% Зачем: Задает стиль библиографии
% Почему: Пункт 2.8.6 Требований по оформлению пояснительной записки.
\bibliographystyle{styles/belarus-specific-utf8gost780u}


% Зачем: Пакет для вставки картинок
% Примечание: Объяснение, зачем final - http://tex.stackexchange.com/questions/11004/why-does-the-image-not-appear
%\usepackage[final]{graphicx}
\usepackage{graphicx}
\DeclareGraphicsExtensions{.pdf,.png,.jpg,.eps}


% Зачем: Директория в которой будет происходить поиск картинок
\graphicspath{{figures/}}

% Зачем: Добавление подписей к рисункам
\usepackage[singlelinecheck=false]{caption}
% \captionsetup[figure]{skip=14pt}
\usepackage{subcaption}


% Зачем: чтобы работала \No в новых латехах
\DeclareRobustCommand{\No}{\ifmmode{\nfss@text{\textnumero}}\else\textnumero\fi}

% Зачем: поворот ячеек таблиц на 90 градусов
\usepackage{rotating}
\DeclareRobustCommand{\povernut}[1]{\begin{sideways}{#1}\end{sideways}}


% Зачем: когда в формулах много кириллических символов команда \text{} занимает много места
\DeclareRobustCommand{\x}[1]{\text{#1}}

% Зачем: Задание подписей, разделителя и нумерации частей рисунков
% Почему: Пункт 2.5.5 Требований по оформлению пояснительной записки.
\DeclareCaptionLabelFormat{stbfigure}{\\Рисунок \arabic{section}.\arabic{figure}}
\DeclareCaptionLabelFormat{stbtable}{Таблица \arabic{section}.\arabic{table}}
\DeclareCaptionLabelSeparator{stb}{~--~}
\captionsetup{labelsep=stb}
\captionsetup[figure]{labelformat=stbfigure,justification=centering,aboveskip=0pt,belowskip=0pt}
\captionsetup[table]{labelformat=stbtable,justification=raggedright,aboveskip=0pt}

% Зачем: Окружения для оформления формул
% Почему: Пункт 2.4.7 требований по оформлению пояснительной записки и специфические требования различных кафедр
% Пример использования смотри в course_content.tex, строка 5
\usepackage{calc}
\newlength{\lengthWordWhere}
\settowidth{\lengthWordWhere}{где}
\newenvironment{explanationx}
    {%
    %%% Следующие строки определяют специфические требования разных редакций стандартов. Раскоменнтируйте нужную строку
    %% стандартный абзац, СТП-01 2010
    %\begin{itemize}[leftmargin=0cm, itemindent=\parindent + \lengthWordWhere + \labelsep, labelsep=\labelsep]
    %% без отступа, СТП-01 2013
    \begin{itemize}[leftmargin=0cm, itemindent=\lengthWordWhere + \labelsep , labelsep=\labelsep]%
    \renewcommand\labelitemi{}%
    }
    {%
    %\\[\parsep]
    \end{itemize}
    }

% Старое окружение для "где". Сохранено для совместимости
\usepackage{tabularx}

\newenvironment{explanation}
    {
    %%% Следующие строки определяют специфические требования разных редакций стандартов. Раскоменнтируйте нужные 2 строки
    %% стандартный абзац, СТП-01 2010
    %\par
    %\tabularx{\textwidth-\fivecharsapprox}{@{}ll@{ --- } X }
    %% без отступа, СТП-01 2013
    \noindent
    \tabularx{\textwidth}{@{}ll@{ --- } X }
    }
    {
    \\[\parsep]
    \endtabularx
    }


% Зачем: Удобная вёрстка многострочных формул, масштабирующийся текст в формулах, формулы в рамках и др
\usepackage{amsmath}


% Зачем: Поддержка ажурного и готического шрифтов
\usepackage{amsfonts}


% Зачем: amsfonts + несколько сотен дополнительных математических символов
\usepackage{amssymb}


% Зачем: Окружения «теорема», «лемма»
\usepackage{amsthm}


% Зачем: Производить арифметические операции во время компиляции TeX файла
\usepackage{calc}

% Зачем: Производить арифметические операции во время компиляции TeX файла
\usepackage{fp}

% Зачем: Пакет для работы с перечислениями
\usepackage{enumitem}
\makeatletter
 \AddEnumerateCounter{\asbuk}{\@asbuk}{щ)}
\makeatother


% Зачем: Устанавливает символ начала простого перечисления
% Почему: Пункт 2.3.5 Требований по оформлению пояснительной записки.
\setlist{nolistsep}


% Зачем: Устанавливает символ начала именованного перечисления
% Почему: Пункт 2.3.8 Требований по оформлению пояснительной записки.
\renewcommand{\labelenumi}{\asbuk{enumi})}
\renewcommand{\labelenumii}{\arabic{enumii})}

% Зачем: Устанавливает отступ от границы документа до символа списка, чтобы этот отступ равнялся отступу параграфа
% Почему: Пункт 2.3.5 Требований по оформлению пояснительной записки.

\setlist[itemize,0]{itemindent=\parindent + 2.2ex,leftmargin=0ex,label=--}
\setlist[enumerate,1]{itemindent=\parindent + 2.7ex,leftmargin=0ex}
\setlist[enumerate,2]{itemindent=\parindent + \parindent - 2.7ex}

% Зачем: Включение номера раздела в номер формулы. Нумерация формул внутри раздела.
\AtBeginDocument{\numberwithin{equation}{section}}

% Зачем: Включение номера раздела в номер таблицы. Нумерация таблиц внутри раздела.
\AtBeginDocument{\numberwithin{table}{section}}

% Зачем: Включение номера раздела в номер рисунка. Нумерация рисунков внутри раздела.
\AtBeginDocument{\numberwithin{figure}{section}}

\let\oldequation=\equation
\let\endoldequation=\endequation
\renewenvironment{equation}{\vspace{-0.3em}\begin{oldequation}}{\end{oldequation}\vspace{-0.9em}}

% Зачем: Дополнительные возможности в форматировании таблиц
\usepackage{makecell}
\usepackage{multirow}
\usepackage{array}


% Зачем: "Умная" запятая в математических формулах. В дробных числах не добавляет пробел
% Почему: В требованиях не нашел, но в русском языке для дробных чисел используется {,} а не {.}
\usepackage{icomma}

% Зачем: макрос для печати римских чисел
\makeatletter
\newcommand{\rmnum}[1]{\romannumeral #1}
\newcommand{\Rmnum}[1]{\expandafter\@slowromancap\romannumeral #1@}
\makeatother


% Зачем: Управление выводом чисел.
\usepackage{sistyle}
\SIdecimalsign{,}

% Зачем: inline-коментирование содержимого.
\newcommand{\ignore}[2]{\hspace{0in}#2}


% Зачем: Возможность коментировать большие участки документа
\usepackage{verbatim}


\usepackage{xcolor}


% Зачем: Оформление листингов кода
% Примечание: final нужен для переопределения режима draft, в котором листинги не выводятся в документ.
\usepackage[final]{listings}

\renewcommand{\lstlistingname}{Листинг}

\lstdefinelanguage{TypeScript}{
    keywords={typeof, number, string, boolean, public, private, protected, virtual, static, override, class, interface, declare, module, any, new, true, false, catch, function, return, null, catch, switch, var, if, in, while, do, else, case, break, for, console, log},
    keywordstyle=\text,
    ndkeywords={class, export,exports, boolean, throw, implements, import, this},
    keywordstyle=\text,
    identifierstyle=\color{black},
    sensitive=false,
    showstringspaces=false,
    comment=[l]{//},
    morecomment=[s]{/*}{*/},
    commentstyle=\color{black}\ttfamily,
    stringstyle=\color{black}\ttfamily,
    morestring=[b]',
    morestring=[b]",
    frame=none,
    numbersep=16pt,
    rulecolor=\color{black},
    breaklines=true,
    columns=fullflexible,
    basicstyle=\normalsize\ttfamily
}

% Зачем: Нумерация листингов в пределах секции
\AtBeginDocument{\numberwithin{lstlisting}{section}}

\usepackage[normalem]{ulem}

\usepackage[final,hidelinks]{hyperref}
% Моноширинный шрифт выглядит визуально больше, чем пропорциональный шрифт, если их размеры одинаковы. Искусственно уменьшаем размер ссылок.
\renewcommand{\UrlFont}{\small\rmfamily\tt}

\usepackage[square,numbers,sort&compress]{natbib}
\setlength{\bibsep}{0em}

% Магия для подсчета разнообразных объектов в документе
\usepackage{lastpage}
\usepackage{totcount}
\regtotcounter{section}

\usepackage{etoolbox}

\newcounter{totfigures}
\newcounter{tottables}
\newcounter{totreferences}
\newcounter{totequation}

\providecommand\totfig{}
\providecommand\tottab{}
\providecommand\totref{}
\providecommand\toteq{}

\makeatletter
\AtEndDocument{%
  \addtocounter{totfigures}{\value{figure}}%
  \addtocounter{tottables}{\value{table}}%
  \addtocounter{totequation}{\value{equation}}
  \immediate\write\@mainaux{%
    \string\gdef\string\totfig{\number\value{totfigures}}%
    \string\gdef\string\tottab{\number\value{tottables}}%
    \string\gdef\string\totref{\number\value{totreferences}}%
    \string\gdef\string\toteq{\number\value{totequation}}%
  }%
}
\makeatother

\pretocmd{\section}{\addtocounter{totfigures}{\value{figure}}\setcounter{figure}{0}}{}{}
\pretocmd{\section}{\addtocounter{tottables}{\value{table}}\setcounter{table}{0}}{}{}
\pretocmd{\section}{\addtocounter{totequation}{\value{equation}}\setcounter{equation}{0}}{}{}
\pretocmd{\bibitem}{\addtocounter{totreferences}{1}}{}{}



% Для оформления таблиц не влязящих на 1 страницу
\usepackage{longtable}

% Для включения pdf документов в результирующий файл
\usepackage{pdfpages}

% Для использования знака градуса и других знаков
% http://ctan.org/pkg/gensymb
\usepackage{gensymb}

% Зачем: преобразовывать текст в верхний регистр командой MakeTextUppercase
\usepackage{textcase}

% Зачем: Переносы в словах с тире.
% Тире в словае заменяем на \hyph: аппаратно\hyphпрограммный.
% https://stackoverflow.com/questions/2193307/how-to-get-latex-to-hyphenate-a-word-that-contains-a-dash#
\def\hyph{-\penalty0\hskip0pt\relax}

% Добавляем абзацный отступ для библиографии
% https://github.com/mstyura/bsuir-diploma-latex/issues/19
\setlength\bibindent{-1.0900cm}

\makeatletter
\renewcommand\NAT@bibsetnum[1]{\settowidth\labelwidth{\@biblabel{#1}}%
   \setlength{\leftmargin}{\bibindent}\addtolength{\leftmargin}{\dimexpr\labelwidth+\labelsep\relax}%
   \setlength{\itemindent}{-\bibindent+\fivecharsapprox-0.240cm}%
   \setlength{\listparindent}{\itemindent}
\setlength{\itemsep}{\bibsep}\setlength{\parsep}{\z@}%
   \ifNAT@openbib
     \addtolength{\leftmargin}{\bibindent}%
     \setlength{\itemindent}{-\bibindent}%
     \setlength{\listparindent}{\itemindent}%
     \setlength{\parsep}{10pt}%
   \fi
}

\setstretch{0.98}

\makeatletter
\renewcommand\@biblabel[1]{#1}
\makeatother

\lstdefinestyle{fsharpstyle}{
   xleftmargin=1.25cm,
   %basicstyle=\footnotesize\ttfamily,
   basicstyle=\fontencoding{T1}\small\ttfamily,
   breaklines=true,
   columns=fullflexible
}

\usepackage{changepage}
\let\OldTexttt\texttt
\renewcommand{\texttt}[1]{\OldTexttt{\fontencoding{T1}\ttfamily{#1}}}

% Приложение
\newcounter{appendix}
\renewcommand*\theappendix{\Asbuk{appendix}}
\newcommand{\appendixtitle}[2]{
    \clearpage\newpage
    \titleformat{name=\section,numberless=true}[block]{}{}{0em}{\centering}  % Should be set only once i guess.
    \refstepcounter{appendix}
    \section*{ПРИЛОЖЕНИЕ \theappendix \\  \centering\textit{(#1)} \\ \centering{#2}}
    \phantomsection                                                                         % Explanation for hyperref.
    \addcontentsline{toc}{section}{ПРИЛОЖЕНИЕ \theappendix}                          % Добавление в toc.
  }
